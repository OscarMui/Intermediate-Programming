\chapter{Data types}

C++ is a strongly typed language, i.e. you have to specify the type of variables when initializing them, and variables could not store data of different types.  

\section{Primary data types}
Primary data types are those that are not composed of other data types.

Basic knowledge: 1 bit is a single 0 or 1 in memory, 8 bits form 1 byte, we usually discuss memory in terms of bytes instead of bits.

\begin{table}[h]
    \centering
    \begin{tabular}{|m{6em}|m{6em}|m{10em}|m{12em}|}
        \hline
        \textbf{Data Type} & 
        Bytes in Memory & 
        Range & 
        Remarks 
        \\ \hline \hline
        
        \texttt{int} &
        4 & 
        $-2^{31}$ to $2^{31}-1$ (i.e. -2147483648 to 2147483647) &
        
        \\ \hline
        
        \texttt{bool} &
        1 & 
        true / false  &
        \tablefootnote{Out of scope: in C++ bool is a primary data type, in C you will have to include stdbool.h to use it.} 
        \\ \hline
        
        \texttt{char} &
        1 & 
        /  &
        Stored using an integer, the ASCII code of the character involved.
        \\ \hline
        
        \texttt{float} &
        4 &
        A large range \textcolor{gray}{ (~$-10^{38}$ to $10^{38}$)} &
        Stores numbers with decimal places
        \\ \hline
        
        \texttt{double} &
        8 & 
        A large range \textcolor{gray}{ (~$-10^{308}$ to $10^{308}$)} &
        Stores numbers with decimal places
        \\ \hline
    \end{tabular}
\end{table}
\pagebreak

These are variants of integers. They are of less importance.\footnote{Reference: \href{https://en.cppreference.com/w/cpp/language/types}{https://en.cppreference.com/w/cpp/language/types}, the range and bytes occupied differs for different version of C++}

\begin{table}[h]
    \centering
    \begin{tabular}{|m{6em}|m{6em}|m{10em}|m{12em}|}
        \hline
        \textbf{Data Type} & 
        Bytes in Memory & 
        Range & 
        Remarks 
        \\ \hline \hline
        
        \texttt{unsigned} &
        4 & 
        $0$ to $2^{32}-1$ (i.e. 0 to 4294967295) &
        Cannot store negative numbers, yet approximately double the range.
        \\ \hline
        
        \texttt{short} &
        2 & 
        $-2^{15}$ to $2^{15}-1$ (i.e. -32768 to 32767) &
        Half the size of an \texttt{int}
        \\ \hline
        
        \texttt{long long} &
        8 & 
        $-2^{63}$ to $2^{63}-1$ &
        Double the size of an \texttt{int}
        \\ \hline
        
    \end{tabular}
\end{table}

\section{Integer \index{overflow}}

As you saw in the last section, integers have a limited range. Overflow occurs when the value we want to store is out of the range that the data type can represent. 

For integers (\texttt{int},\texttt{unsigned},\texttt{short},\texttt{long long}), the value it represents will go to the other end of the range when they overflow.\footnote{Reason behind out of scope: requires understanding of two's compliment}. 

For example,
\begin{lstlisting}
int x = 2147483647;
cout << "Before overflow: " << x << endl; //2147483647
x += 1;
cout << "After overflow: " << x << endl; //-2147483648

unsigned y = 4294967295;
cout << "Before overflow: " << y << endl; //4294967295
y += 1;
cout << "After overflow: " << y << endl; //0
\end{lstlisting}

\section{Floating point values}

OK, the range of floating point values (\texttt{float} and \texttt{double}) are much larger than ints, so is it a good idea to use them all the time? NO!

Floating point values store values inaccurately, and rounding errors accumulate. Even though using \texttt{double} reduces the rounding error, floating point values are still nasty in some ways. 

You are advised to only use \texttt{double} to store numbers with decimal points and very large numbers (those can't be represented by \texttt{long long}). And never use \texttt{float}\footnote{Memory usage is not a problem nowadays}.

\section{Boolean operators}

A \index{Boolean} variable (\texttt{bool}) can be either \texttt{true} or \texttt{false}. There are three operators that we can perform on \texttt{bool}s. Including AND (\&\&), OR (\textbar\textbar), NOT (!).

\begin{table}[h]
    \centering
    \begin{tabular}{|m{4em}|m{4em}|m{4em}|m{4em}|m{4em}|}
        \hline
        a & 
        b & 
        a \&\& b & 
        a\textbar\textbar b & 
        !a 
        \\ \hline \hline
        
        true & 
        true & 
        true & 
        true & 
        false 
        \\ \hline
        
        true & 
        false & 
        false & 
        true & 
        false 
        \\ \hline
        
        false & 
        true & 
        false & 
        true & 
        true 
        \\ \hline
        
        false & 
        false & 
        false & 
        false & 
        true 
        \\ \hline
        
    \end{tabular}
\end{table}

C/C++ recognizes 0 as false and other non-zero values as true. (using 1 is usually the convention) It is common to use 0 instead of false and 1 instead of true.

You could do \texttt{while(true) ...} for an infinite loop, but you might as well do \texttt{while(1) ...}
\vspace{6mm}

Order of precedence: ! $>$ \&\& $>$ \textbar\textbar

For example, 

!0 \textbar\textbar 0 \&\& 1

= (!0) \textbar\textbar 0 \&\& 1

= 1 \textbar\textbar 0 \&\& 1

= 1 \textbar\textbar (0 \&\& 1)

= 1 \textbar\textbar 0

= 1

\subsection{Bitwise operators}

Don't have time to cover, there should be plenty of resources online on this topic. 

Yet, it is important for senior group contestants as this kind of questions seem common. 

\includegraphics[width=7cm]{ch3-bitwise.png}
\footnote{Reference: \href{https://www.digitalocean.com/community/tutorials/python-bitwise-operators}{https://www.digitalocean.com/community/tutorials/python-bitwise-operators}}

\section{Characters are integers}

Let's run some C++ code.

\begin{lstlisting}
char a = 'a'; //we use single quotes for characters and double quotes for strings
printf("%d %c\n",a,a); //97 a
a = 97;
printf("%d %c\n",a,a); //97 a
a = 65;
printf("%d %c\n",a,a); //65 A
a += 1;
printf("%d %c\n",a,a); //66 B
\end{lstlisting}

It seems that each character is associated with an integer. That's right! 128 (later extended to 256) frequently used characters are chosen and they are associated with an integer between 0 and 127.\footnote{ASCII Table: \href{https://simple.wikipedia.org/wiki/ASCII}{https://simple.wikipedia.org/wiki/ASCII}}

We can perform arithmetic operations with \texttt{char} variables, as they are just integers! 

\section{Strings}

Strings are just an array of characters, but there are a few nasty catches.

\subsection*{Strings terminate with a null character \texttt{\textbackslash 0}}

Let's think about what \texttt{char s[10] = "Hello";} does. It is actually equivalent to 

\includegraphics[width=16cm]{ch3-nullstring.png}

The \texttt{\textbackslash 0} (ASCII code = 0) indicates the end of a string, it is necessary for C/C++ programs so that they could identify when the string ends. If you remove it by hand something bad would happen...

Don't trust me? Try running:

\begin{lstlisting}
char s[4] = "FGH"; //without the terminating character
s[3] = 'I';
cout << s << endl; //FGHI followed by a bunch of nonsense, the nonsense is different everytime we run the program

char s1[7] = { 'A', 'B', 'C', '\0', 'D', 'E', '\0'};
cout << s1 << endl; //ABC
\end{lstlisting}

\includegraphics[width=3cm]{ch3-beyondnull.png}

The first example shows what happens when we explicitly remove the null character at the end of the string. (By replacing \texttt{FGH\textbackslash 0} with \text{FGHI}) It attempts to access and print things beyond the memory allocated. (see chapter 5 for more details)

The second example verifies that the null character is recognized by C/C++ as the end of the string, despite having some other contents after the null character.

\textbf{So we always have to reserve one more space for the null character.}

\subsection*{\texttt{cin} obtains input word by word by default}

\begin{lstlisting}
char s2[20];
cin >> s2; //You input Hello world
cout << s2 << endl; //Hello
\end{lstlisting}

This example shows that \texttt{cin} only catches the first word and store it in s2, while leaving other words for another variable.

\begin{lstlisting}
char s2[20], s3[20];
cin >> s2; //You input Hello world
cout << s2 << endl; //Hello

cin >> s3; //The computer did not ask for your input
cout << s3 << endl; //world
\end{lstlisting}

However, sometimes you want to obtain a whole line of text input rather than just a word. You could use \texttt{cin.getline}. It accepts two arguments, first the string variable (the character array you defined), second the maximum amount of characters it should get (you have to set it equal to the size of the string)

\begin{lstlisting}
char s4[20];
cin.getline(s4,20); //You input Good morning
cout << s4 << endl; //Good morning
\end{lstlisting}

Unfortunately it is not as simple as that, things get complicated when you have other \texttt{cin}s in the same program.

\begin{lstlisting}
char s2[20], s3[20];
cin >> s2; //You input Hello world
cout << s2 << endl; //Hello

cin >> s3; //The computer did not ask for your input
cout << s3 << endl; //world

char s4[20];
cin.getline(s4,20); //The computer did not ask for your input
cout << s4 << endl; //Probably a next line symbol
\end{lstlisting}

This behaviour is probably\footnote{Who is certain about this chaos?} due to the next line symbol of the Hello world line has not been read, and \texttt{cin.getline} reads the text until it hits a next line symbol, so the only thing stored in s4 is the next line symbol...

What you could do is use \texttt{cin.clear} and \texttt{cin.ignore(10000,'\textbackslash n'} before \texttt{cin.getline}, in order to clear the buffer.

\begin{lstlisting}
char s2[20], s3[20];
cin >> s2; //You input Hello world
cout << s2 << endl; //Hello

cin >> s3; //The computer did not ask for your input
cout << s3 << endl; //world

char s4[20];
cin.clear();
cin.ignore(10000, '\n');
cin.getline(s4,20); //You input Good morning
cout << s4 << endl; //Good morning
\end{lstlisting}

Alternatively, you could call \texttt{cin.getline} twice, the first one to read the previous line, and the second one to actually read your input. The result of the first read is overwritten by the second one as desired.

\begin{lstlisting}
char s2[20], s3[20];
cin >> s2; //You input Hello world
cout << s2 << endl; //Hello

cin >> s3; //The computer did not ask for your input
cout << s3 << endl; //world

char s4[20];
cin.getline(s4,20); //read the previous line
cin.getline(s4,20); //You input Good morning
cout << s4 << endl; //Good morning
\end{lstlisting}

\subsection{Strings VS string variables}

The following two restrictions concern string variables but not strings. Now let's differentiate strings and string variables.

\subsection*{String variables cannot be compared normally}

Strings are only for one time use. You can compare them normally using comparison operators.

\begin{lstlisting}
cout << ("ABC"=="DEF") << endl; //0
cout << ("ABC" < "DEF") << endl; //1 (why?)
cout << ("abc" <= "ABC") << endl; //0 (why?)
\end{lstlisting}

String variables are character arrays that are used to store strings. Sadly they cannot be compared normally. (use strcmp instead)\footnote{Reason out of scope: because the references of the arrays are compared instead of their values. You may understand more in Chapter 5 but the full concept is not necessary}

\begin{lstlisting}
char s5[20] = "ABC", s6[20] = "ABC", s7[20] = "DEF";
cout << (s5==s5) << endl; //1
cout << (s5==s6) << endl; //0
cout << (s5<s6) << endl; //0
\end{lstlisting}

\subsection*{The whole string variable cannot be reassigned}

Thought you could do this? Nah...

\begin{lstlisting}
char s8[20] = "ABCD";
s8 = "XYZ"; //SYNTAX ERROR
\end{lstlisting}

Well you could do the following instead if you are desperate, but there is a better way in the next section. (strcpy)

\begin{lstlisting}
s8[0] = 'X'; 
s8[1] = 'Y'; 
s8[2] = 'Z'; 
s8[3] = '\0'; 
\end{lstlisting}


\subsection{Operation on string variables (IMPORTANT)}

\begin{table}[h]
    \centering
    \begin{tabular}{|m{10em}|m{25em}|}
        \hline
        \textbf{Function} & 
        Usage 
        \\ \hline \hline
        
        \texttt{strlen(s)} &
        Get string length
        \\ \hline
        
        \texttt{strcpy(s1, s2)} &
        Copy value stored in \texttt{s2} to \texttt{s1} 
        \\ \hline
        
        \texttt{strcat(s1, s2)} &
        Concatenate (i.e. combine) the two strings, store it in \texttt{s1} 
        \\ \hline
        
        \texttt{strcmp(s1, s2)} &
        Compare two strings, returns 0 if \texttt{s1} = \texttt{s2}, returns a negative number if \texttt{s1} $<$ \texttt{s2}, returns a positive number if \texttt{s1} $>$ \texttt{s2}. (In dictionary order with respect to ASCII code)
        \\ \hline
    \end{tabular}
\end{table}

You are advised to try out these operations yourself, I am sure you will find surprises.

\section{Conclusion and further resources}

\href{https://www.youtube.com/watch?v=Zy2bNkSxv8M}{This video}\footnote{Link: \href{https://www.youtube.com/watch?v=Zy2bNkSxv8M}{https://www.youtube.com/watch?v=Zy2bNkSxv8M}} summarizes some more nasty things about strings. Bear in mind what you can and what you can't do with them.
\vspace{6mm}

\begin{center}
\textit{The most comfortable type to deal with are integers.}
\end{center}
