\documentclass[12pt]{ociamthesis}  % default square logo 
% \documentclass[12pt,beltcrest]{ociamthesis} % use old belt crest logo
%\documentclass[12pt,shieldcrest]{ociamthesis} % use older shield crest logo

%load any additional packages

% ---- START OF CUSTOM CONFIGURATION ----
\def\proglang{0} %1: Java, others C++
\def\invariant{1} %1: include contents on Invariants, others otherwise (for chapter 7 only)
\def\aians{0} %1: contains answer to questions given by ChatGPT

\usepackage{amssymb} % unknown
\usepackage{array} % unknown
\usepackage{courier} % give courier font to listings and also \texttt{}
% \usepackage{bera} % Alternative font, still in testing
\usepackage{xcolor} % add colors to text using \textcolor{red}{<text>}
\usepackage{amsmath} % unknown
\usepackage{ulem} % unknown
\usepackage{enumitem} % more sophisticated numbering system for enumerate
\usepackage{hyperref} % ability to add links using \href{<link>}{<label>}, also ability to make the contents page clickable
\usepackage{listings} % ability to print code using \begin{lstlisting}
\usepackage{graphicx} % ability to add image
\usepackage{tablefootnote} % ability to use table footnotes in table using command \tablefootnotes
\usepackage{multirow} % ability to merge rows using \mergerow
\usepackage{makecell} % ability to have line breaks inside tables
\usepackage{xurl} % Better url line breaks
\usepackage{imakeidx} % add an index for the document
\usepackage{cleveref} % Create references to different chapters
\usepackage{parskip} % Adding space between paragraphs without affecting other aspects like table of contents, also removes paragraph indents
\usepackage{endnotes} % For displaying solutions

% Configure solutions
% \def\enotesize{\normalsize}
% \def\makeenmark{\relax}
% \def\notesname{Solutions}
% \def\answer#1{\endnotetext{#1}}
% \def\theanswers{\theendnotes \medskip}

\makeindex % add an index for the document
% configure hyperref, display the contents page as its usual black color
\hypersetup{ 
    colorlinks,
    citecolor=black,
    filecolor=black,
    linkcolor=black,
    urlcolor=black
}

% configure listings

\if\proglang1
\lstset{language=Java}
\else
\lstset{language=C++}
\fi
\lstset{
    basicstyle=\ttfamily,
    breaklines=true,
    keepspaces=true,
    showspaces=false,                
    showstringspaces=false,
} %use courier as font, also make sure it skips to another line when the text is too long for one line

% \setlength{\parindent}{0pt} % disable indent of first line of each paragraph
\graphicspath{ {./images/} } % configure graphicx

%  Configure cleveref
\crefformat{section}{\S#2#1#3} % see manual of cleveref, section 8.2.1
\crefformat{subsection}{\S#2#1#3}
\crefformat{subsubsection}{\S#2#1#3}

% add signature
\pagestyle{myheadings}
\markright{O.M. - Intermediate Programming}

% custom environments and comments
\newenvironment{questions}
    {
        \begin{enumerate}[label=\textbf{\thechapter.\arabic*}]
    }
    { 
        \end{enumerate}
    }

\newenvironment{parts}
    {
        \begin{enumerate}[label=(\alph*)]
        \let\part\mipart
    }
    { 
        \end{enumerate}
    }

\newenvironment{multiplechoice}
    {   
        \begin{samepage}
        \nopagebreak % no breaking between choices
        \begin{enumerate}[label=\Alph*.]
    }
    {
        \end{enumerate}
        \end{samepage}
    }

% revised question command that tries to encourage page breaks
% to lie between questions rather than within questions
% Copied from oxford problem sheet template
% \newcommand{\miquestion}[1][]{\filbreak
%   \ifthenelse{\equal{#1}{}}{\item}{\item[#1]}
% }

\newcommand{\miquestion}{\item}

\newcommand{\mipart}{\item}

\newcommand{\link}[2]{\href{#1}{#2}\footnote{Link: \url{#1}}}

\newcommand{\footnotecite}[1]{{\footnotesize \cite{#1}}}

\newcommand{\code}[1]{\texttt{#1}}

\newcommand{\answer}[1]{\endnotetext{#1}}

% ---- END OF CUSTOM CONFIGURATION ----

%input macros (i.e. write your own macros file called mymacros.tex 
%and uncomment the next line)
%\include{mymacros}

\title{Intermediate Programming %your thesis title,
        }   %note \\[1ex] is a line break in the title

\author{Oscar Mui }             %your name
\college{University College}  %your college

% \renewcommand{Notes for}{change the default text here if needed}
\degree{Computer Science \\[1ex] Second Edition (in C++)}     %the degree
\degreedate{August 2023}         %the degree date

% \includeonly{chapter1,chapter1ex,solutions}

%end the preamble and start the document
\begin{document}

%this baselineskip gives sufficient line spacing for an examiner to easily
%markup the thesis with comments
\baselineskip=18pt plus1pt

%set the number of sectioning levels that get number and appear in the contents
\setcounter{secnumdepth}{3}
\setcounter{tocdepth}{3}

\maketitle                  % create a title page from the preamble info

\begin{romanpages}          % start roman page numbering

\include{dedication}        % include a dedication.tex file
\include{acknowlegements}   % include an acknowledgements.tex file
\chapter*{Preface}

This piece of notes covers some topics in the \href{https://hkoi.org/en/competition-syllabus/}{Hong Kong Olympiad in Informatics (HKOI) syllabus} \footnote{Link: \href{https://hkoi.org/en/competition-syllabus/}{https://hkoi.org/en/competition-syllabus/}} that I found the most useful. Aiming to introduce senior secondary school students with more advanced programming theories, so as to prepare them for programming competitions like HKOI. It also serves as a good revision material for DSE ICT coding module (however not everything in the DSE syllabus is covered).


It is a very simplified piece of notes with a huge number of links to other resources, as the internet is a better teacher than me, yet I am here to provide you with information that I found the most useful when I was in your position a few years ago.

\section*{About me}

I am a second-year undergraduate student studying Computer Science at the University of Oxford, United Kingdom. I joined the senior group HKOI before in the year 2021-2022 and obtained a silver award at the finals. 

\section*{A word of warning}

As you see in the title, this is just a draft, aiming to include everything in the shortest amount of time possible, so explanations and examples may be inadequate. If there are any errors in the notes feel free to contact me by email oscar.mui@univ.ox.ac.uk

          % include the abstract


\tableofcontents            % generate and include a table of contents
% \listoffigures              % generate and include a list of figures
\end{romanpages}            % end roman page numbering

%now include the files of latex for each of the chapters etc
\include{chapter0}
% Translated to Java
\chapter{Elementary Programming}
\label{sec:elementary}

Here are some basic programming knowledge. I hope you have seen some of the concepts in this chapter already on other occasions. This is a good opportunity to recap, or to familiarise yourself with C++ if you are not a C++ programmer. I would like to stress that most programming concepts discussed in this piece of notes are transferable to most other programming languages.

% Translated to Java end 

\section*{Further resources (Ch 1-2)}
\addcontentsline{toc}{section}{Further resources (Ch 1-2)}
\href{https://www.youtube.com/watch?v=tvC1WCdV1XU&list=PLAE85DE8440AA6B83}{Bucky's C++ Programming Tutorial}\footnote{Link: \url{https://www.youtube.com/watch?v=tvC1WCdV1XU&list=PLAE85DE8440AA6B83}} (a YouTube playlist) covers most things that you need to know about C++, and also most of the things in this chapter. You only need to watch the first 20-30 videos, as it goes too deep in the later episodes.

\section*{Practicing at home}
Practicing is very important. For example, one of the things I would do back then is to remove the one or two lines that I didn't understand in the materials, and see how they affected the program by printing out the values of the variables at different times. 

Refer to \cref{sec:settingup} on how to run C++ code on your own device.

\section{Structure of a C++ program}
\begin{lstlisting}
//First include the libraries that you are going to use
#include <iostream> 

//A weird line of code that you have to remember every time you write a C++ program.
using namespace std;

int main(){
    cout << "Hello world" << endl;
    return 0;
}
\end{lstlisting}

The main function is the point of entry of the program.

You need to add a semicolon at the end of every statement or else the compiler will shout at you. 

Comments are denoted as \texttt{//}.

\section{Variables}

\textit{A variable is just like an empty box. You can store stuff in it. Technically, it is nothing but a name given to a computer memory. That's it, now go spend rest of your life deciding the perfect variable name.}\footnote{Cited from \cite{variableig}}

It is important to distinguish between initialision of the variable and reassignment of the variable.

\begin{lstlisting}
int score = 70; //initialisation
score = 80; //reassignment
\end{lstlisting}

You have to specify the type of variables when initialising them in C++. Each variable can only be used to store data of a single type. You can see a list of datatypes that C++ offers in \cref{sec:primarydtypes} and we will dedicate our full attention to them in \cref{sec:dtypes}.

It is not compulsory to give an initial value to a variable. If you do not do so, it will be a random number until you reassigns it.

\begin{lstlisting}
int score = 70; //initialisation
cout << score << endl; //unexpected value
score = 80; //reassignment
cout << score << endl; //80
\end{lstlisting}

\section{Console I/O}

\texttt{cout} stands for Console OUTput. You put the things (variable or strings - sequence of characters) that you want to print after the \texttt{cout} keyword, separated by \texttt{<<}. The \texttt{endl} keyword inserts a newline character. Remember to use double quotes for strings.

\texttt{cin} stands for Console INput. You put variables that you would like the output to be stored in after the \texttt{cin} keyword, separated by \texttt{>>}. 

The direction of the arrows may be hard to remember, but it is intuitive once you associate them with the direction of information flow. You want text to flow towards the screen (the \texttt{cout} keyword) when you are outputting, and towards the variables when asking for user input.

\begin{lstlisting}
cout << "What is your score? ";
int score;
cin >> score;
cout << "Your scored " << score << " in this paper." << endl;
\end{lstlisting}

\section{Conditionals}
\subsection{\texttt{if} statements}

\begin{lstlisting}
int score;
cin >> score;
if(score >= 70){
    printf("Good job.\n");
}else if(score >= 40){
    printf("You got a pass.\n");
}else{
    printf("You failed.\n");
}
\end{lstlisting}

\subsection{\texttt{switch case}}

Just looks neater when you are testing on the same variable multiple times. You can of course use if else if else if... 

Note that it only works for int and char, and only equality tests are allowed. For example, the above example on scores could not be replaced using switch case.

Also note that the \textbf{\texttt{break}} keyword is necessary to quit the switch statement, or else it will run the default clause after the \texttt{case} clause.

\begin{lstlisting}
char x;
cin >> x;
switch(x){
    case 'z':
        cout << "It is the last letter of the alphabet." << endl;
        break;
    //this is how you do multiple equality tests
    case 'a':
    case 'e':
    case 'i':
    case 'o':
    case 'u':
        cout << "It is a vowel." << endl;
        break;
    //equivalent to the else clause
    default:
        cout << "It is not a vowel." << endl;
        break;
}
\end{lstlisting}

% Translated to Java

\section{Arrays}
\label{sec:arrayintro}
Arrays store sequences of data of the same type.

\if\proglang1
\begin{lstlisting}
int[] x = {3,1,4,1,5,9,2,6};

System.out.println(x[0]); //3 
System.out.println(x[4]); //5
System.out.println(x[7]); //6 (last element)
System.out.println(x[8]); //Error 
\end{lstlisting}
\else
\begin{lstlisting}
int x[8] = {3,1,4,1,5,9,2,6};
//alternatively: int x[] = {3,1,4,1,5,9,2,6}; length of array can be omitted as the compiler can derive it from the right hand side

cout << x[0] << endl; //3 
cout << x[4] << endl; //5
cout << x[7] << endl; //6 (last element)
cout << x[8] << endl; //unexpected value (Why?)
\end{lstlisting}
\fi

We call the things stored in the array \textbf{elements}. We distinguish different elements using \textbf{index}. The index \textbf{starts from 0}, and ends at $n-1$, where $n$ is the length of the array. The \textbf{length} of the array refers to the maximum number of elements that the array can store, based on the memory allocated when it is \textbf{initialised}. We cannot change the length of the array normally.

\includegraphics[width=12cm]{images/ch2-arrayindex.png}
\vspace{6mm}

There are two ways of initialising arrays. The first way is to hard-code the elements in the array. The length of the array is inferred by the number of elements given, and it cannot be extended afterwards.

\if\proglang1
\begin{lstlisting}
int[] x = {3,1,4,1,5,9,2,6};
\end{lstlisting}
\else
\begin{lstlisting}
int x[] = {3,1,4,1,5,9,2,6};
\end{lstlisting}
\fi

The second way is when you are not quite sure what the elements of the array are yet, but you must still supply the length of the array.

\if\proglang1
\begin{lstlisting}
int[] x = new int[8];
\end{lstlisting}
\else
\begin{lstlisting}
int x[8];
\end{lstlisting}
\fi

% Translated to Java end

\section{Loops}
\subsection{\texttt{for} loops}
Runs the body a specified number of times.

\begin{lstlisting}
int x[8] = {3,1,4,1,5,9,2,6};
int sum = 0;
for(int i = 0; i < 8; i++) { //loops i=0,1,2,3,4,5,6,7
    sum += x[i];
}
cout << sum << endl; //31
\end{lstlisting}

\begin{lstlisting}
int x[8] = {3,1,4,1,5,9,2,6};
int sum = 0;
for(int i = 7; i >= 0; i--) { //loops i=7,6,5,4,3,2,1,0
    sum += x[i];
}
cout << sum << endl; //31
\end{lstlisting}

\subsection{\texttt{while} loops}
Runs the body until the test is false.
\begin{lstlisting}
int x[8] = {3,1,4,1,5,9,2,6};
int sum = 0;
int i = 0;
while(i<8) { 
    sum += x[i]; //loops i=0,1,2,3,4,5,6,7
    i++;
}
cout << sum << endl; //31
\end{lstlisting}

\begin{lstlisting}
int x[8] = {3,1,4,1,5,9,2,6};
int sum = 0;
int i = 0;
while(i<8&&sum<10) { 
    sum += x[i];
    i++;
}
cout << sum << endl; //14
\end{lstlisting}

\subsection{\texttt{do while} loops}
Runs the body until the test is false. The body will run at least once.

\begin{lstlisting}
bool emergency = false;
do{
    printf("EMERGENCY\n"); //will be printed
}while(emergency);
\end{lstlisting}

\begin{lstlisting}
bool emergency = false;
while(emergency){
    printf("EMERGENCY\n"); //will not be printed
}
\end{lstlisting}

This example yields the same result with while loops because the test is true initially, allowing the loop to run at least once.

\begin{lstlisting}
int x[8] = {3,1,4,1,5,9,2,6};
int sum = 0;
int i = 0;
do { 
    sum += x[i]; //loops i=0,1,2,3,4,5,6,7
    i++;
}while(i<8);
cout << sum << endl; //31
\end{lstlisting}

\subsection{Infinite loops}

If the test case is always true (there is nothing in the loop to make the test false). The loop is an infinite loop\index{infinite loop}, where the loop will not terminate (until resources are used up), and the remaining parts of the program can never be run. In this example, EMERGENCY will be printed forever nonstop.

\begin{lstlisting}
bool emergency = true;
while(emergency){
    printf("EMERGENCY\n");
}
destroy_world(); //will not be run
\end{lstlisting}
\vspace{6mm}

\subsection{\texttt{break}}

\texttt{break} allows you to terminate the loop early.

\begin{lstlisting}
int x[8] = {3,1,4,1,5,9,2,6};
int sum = 0;
int i = 0;
while(i<8) { 
    sum += x[i];
    i++;
    if(sum >= 10) break;
}
cout << sum << endl; //14
\end{lstlisting}

\subsection{\texttt{continue}}

\texttt{continue} allows you to jump to the next iteration, skipping the rest of the current iteration early.

\begin{lstlisting}
int x[8] = {3,1,4,1,5,9,2,6};
int sum = 0;
int i = 8;
while(i>0) {
    i--;
    if(x[i] >= 5) continue;
    sum += x[i];
}
cout << sum << endl; //11
\end{lstlisting}

\subsection{Necessity of \texttt{break} and \texttt{continue}}

% Note that all for loops and do while loops can be written as a while loop.

% \begin{lstlisting}[basicstyle=\rmfamily]
% for(int i = 0; i < <target>; i++) <body> === 
% int i = 0; while(i < <target>) <body> i++;
% \end{lstlisting}

% \begin{lstlisting}[basicstyle=\rmfamily]
% do <body> while(<test>); === <body> while(<test>) <body>
% \end{lstlisting}

Note that the use of \texttt{break} and \texttt{continue} can be avoided by modifying the loop condition and using if statements respectively. 

In fact, plenty of other programming languages do not have \texttt{break} and \texttt{continue}. 

\section{2D arrays}

Don't have time to cover, there should be plenty of resources online on this topic.

\begin{lstlisting}
int x[3][] = {{1,4,7},{2,5,8},{3,6,9}};
cout << x[2][1] << endl; //6
\end{lstlisting}

\subsection{Nested loops}

\begin{lstlisting}
int x[3][] = {{1,4,7},{2,5,8},{3,6,9}};
sum = 0;
for(int i=0;i<3;i++)
    for(int j=0;j<3;j++)
        sum += x[i][j];
cout << sum << endl; //45
\end{lstlisting}

% Translated to Java 
\section{\if\proglang1 Methods \else ~Functions \fi}
\label{sec:functions}

\if\proglang1 Methods \else Functions \fi let you organize code better and reduce the amount of repeated code. You can define them by:

\if\proglang1
\begin{lstlisting}[language=,basicstyle=\rmfamily]
public static <return type> <name>(<arguments>){
    return <return value>;
}
\end{lstlisting}
\else
\begin{lstlisting}[language=,basicstyle=\rmfamily]
<return type> function(<arguments>){
    return <return value>;
}
\end{lstlisting}
\fi

For example, this \texttt{fact}\if\proglang1 method \else ~function \fi accepts an integer x, and returns another integer.
\if\proglang1
\begin{lstlisting}
public static int fact(int x){
    int y = 1;
    for(int i = 1; i <= x; i++){
        y *= i;
    }
    return y;
}
\end{lstlisting}
\else
\begin{lstlisting}
int fact(int x){
    int y = 1;
    for(int i = 1; i <= x; i++){
        y *= i;
    }
    return y;
}
//What would happen if we input a negative number?
\end{lstlisting}
\fi

This is how you use a\if\proglang1 method \else ~function \fi: 
You could call it by just supplying the argument.

\if\proglang1
\begin{lstlisting}
    System.out.println(fact(6)); //720
\end{lstlisting}
\else
\begin{lstlisting}
    cout << fact(6) << endl; //720
\end{lstlisting}
\fi

Alternatively, we usually would prefer saving the value returned for later use.

\begin{lstlisting}
    int x = fact(6); //720 is now stored in x
\end{lstlisting}

You could use the \texttt{\textbf{void}} keyword to indicate that a\if\proglang1 method \else ~function \fi does not have a return value. For \texttt{void}\if\proglang1 methods \else ~functions \fi, if there is no explicit \texttt{return;} statement, the\if\proglang1 method \else ~function \fi will automatically quit at the end of the\if\proglang1 method \else ~function \fi. 

\if\proglang1
\begin{lstlisting}
import java.util.Scanner;

public static void giveComment(int score){
    if(score >= 70){
        System.out.println("Good job.");
    }else if(score >= 40){
        System.out.println("You got a pass.");
    }else{
        System.out.println("You failed.");
    }
    //automatically quits here
}

public static void main(String args[]){
    Scanner scanner = new Scanner(System.in);  
    int s = scanner.nextInt();
    giveComment(s);
}
\end{lstlisting}
\else
\begin{lstlisting}
void giveComment(int score){
    if(score >= 70){
        printf("Good job.\n");
    }else if(score >= 40){
        printf("You got a pass.\n");
    }else{
        printf("You failed.\n");
    }
    //automatically quits here
}

int main(){
    int s;
    cin >> s;
    giveComment(s);
    //returning 0 is optional for some C/C++ compilers, it is implied it returns 0 normally
}
\end{lstlisting}
\fi

The main\if\proglang1 method \else ~function \fi is just a\if\proglang1 method \else ~function \fi with no arguments and an integer as a return value. The main\if\proglang1 method \else ~function \fi (specially recognized by the compiler) is the point of entry of the program, the program terminates when the main program returns, returning 0 indicates that there is no error, and returning other integers indicate otherwise.
\vspace{6mm}

Note that the giveComment\if\proglang1 method \else ~function \fi must be placed in front of the main\if\proglang1 method \else ~function \fi, or else the main\if\proglang1 method \else ~function \fi could not call the giveComment\if\proglang1 method \else ~function \fi as the\if\proglang1 method \else ~function \fi has not been defined by the time the compiler reads till the main\if\proglang1 method \else ~function \fi. 


% TODO: https://users.cs.utah.edu/~zachary/computing/lessons/uces-10/uces-10/node11.html
% Translated to Java end
\section*{Exercises}
\addcontentsline{toc}{section}{Exercises}

\textit{It is recommended that you first do the questions using pen and paper, before trying to run any code on a computer.}

\begin{questions}

\subsection*{Conditionals and Functions}

\miquestion {\footnotesize [A]} Refer to the following code snippet.

\begin{lstlisting}
int score;
cin >> score;
if(score < 70){
    cout << "A";
}
if(score > 40){
    cout << "B";
}else if(score < 60){
    cout << "C";
}else{
    cout << "D";
}
\end{lstlisting}

Determine the output on the following inputs.

\begin{parts}
\part 30
\part 40
\part 60
\part 70
\end{parts}

\label{q:if1}
\answer{ \ref{q:if1}
\begin{parts}
\part AC
\part AC
\part AB
\part B
\end{parts}
}

\miquestion {\footnotesize DSE\cite{dse:2016p1}} Which of the following pseudocodes produce the same result? 

\begin{enumerate}[label=(\arabic*)]
\item 
\begin{lstlisting}[language=Caml]
if P < 3 and Q > 25 then
    R = R + 1
\end{lstlisting}

\item 
\begin{lstlisting}[language=Caml]
if Q > 25 then
    if P < 3 then
        R = R + 1
\end{lstlisting}

\item 
\begin{lstlisting}[language=Caml]
if P < 3 then
    if Q > 25 then
        R = R + 1
\end{lstlisting}
\end{enumerate}

\begin{multiplechoice}
    \item (1) and (2) only
    \item (1) and (3) only
    \item (2) and (3) only
    \item (1), (2) and (3)
\end{multiplechoice}
\label{q:dse:2016p1:q30}

\answer{
    \ref{q:dse:2016p1:q30} D
    \if\aians1
    In logical operations, such as the ``and'' operator, the order of the conditions does not affect the result.

    Therefore, all three pseudocodes (i), (ii), and (iii) are equivalent and produce the same result. They all check if P is less than 3 and Q is greater than 25, and if both conditions are true, they increment R by 1.
    \fi
}

\miquestion {\footnotesize (UNI)} Consider the general form of an nested if statement with three branches, where cond$_1$ and cond$_2$ are boolean expressions that evaluates to true or false, while body$_1$, body$_2$ and body$_3$ are sequences of statements.

\begin{lstlisting}[language=Caml,mathescape=true]
if cond$_1$ then
    body$_1$
else if cond$_2$ then
    body$_2$
else
    body$_3$
\end{lstlisting}

Describe precisely the conditions when body$_1$, body$_2$ and body$_3$ are executed in terms of cond$_1$ and cond$_2$.

Generalise this to $n-1$ conditions and $n$ bodies.

\miquestion {\footnotesize (Of less importance)[A]} Determine the output of the following code snippets.

\begin{parts}
\item 
\begin{lstlisting}
cout << 17/4 << endl;
\end{lstlisting}

\item 
\begin{lstlisting}
cout << 17.0/4 << endl;
\end{lstlisting}

\item 
\begin{lstlisting}
cout << 17/4.0 << endl;
\end{lstlisting}

\item 
\begin{lstlisting}
double a = 17/4;
cout << a << endl;
\end{lstlisting}

\item 
\begin{lstlisting}
int b = 17.0/4;
cout << b << endl;
\end{lstlisting}

\item 
\begin{lstlisting}
double c = 17.0/4.0;
cout << c << endl;
\end{lstlisting}
\end{parts}

\label{q:mod}
\answer{ \ref{q:mod}

\begin{parts}
\item 4
\item 4.25
\item 4.25
\item 4
\item 4
\item 4.25
\end{parts}
}
\miquestion \textbf{(Leap year).} Write a function that takes in a year and determines whether it is a leap year.

\begin{lstlisting}
bool isLeapYear(int year)
\end{lstlisting}

A year is a leap year if it is a multiple of 4. (e.g. 2016,2020,2024 are leap years) \\
However, a year is not a leap year if it is a multiple of 100, even though it is a multiple of 4. (e.g. 2100, 1900 are not leap years) \\
However, a year is a leap year if it is a multiple of 400, even though it is a multiple of 100. (e.g. 1600, 2000 are leap years)

Do you think your answer is the simplest implementation? Discuss with your instructor.

\subsection*{Loops and Functions}
\miquestion {\footnotesize DSE\cite{dse:2016p1}} What is the output of the following code snippet?

% \begin{lstlisting}
% int S = 0;
% for(int j = 1; j <= 5; j++){
%     cout << j;
%     S += j;
% }
% cout << S;
% \end{lstlisting}

\begin{lstlisting}
S <- 0
for J from 1 to 5
    Output J
    S <- S + J
Output S
\end{lstlisting}

\begin{multiplechoice}
\item 12345
\item 1234515
\item 5432115
\item 5432121
\end{multiplechoice}

\label{q:dse:2016p1:q31}

\answer{
    \ref{q:dse:2016p1:q31} B

    \if\aians1
The for loop iterates from j = 1 to j = 5.
Inside the loop, cout << j; prints the value of j on each iteration, resulting in the output "12345".

Additionally, S += j; adds the value of j to the variable S on each iteration, accumulating the sum of the numbers from 1 to 5.
After the loop, cout << S; prints the value of S, which is the sum of the numbers from 1 to 5, resulting in the output "15".

Therefore, the overall output of the code snippet is "1234515".
    \fi
}

\miquestion {\footnotesize HJ\cite{hkoi:2022hj}} What is the output of the following program?

\begin{lstlisting}
int x, y, z, i;
int main() {
    x = 0;
    y = 0;
    z = 0;
    for (i = 1; i <= 2022; i++) {
        if (i % 10 == 0)
            x++;
        else if (i % 5 == 0)
            y++;
        else
            z++;
    }
    cout << x << ' ' << y << ' ' << z;
return 0;
}
\end{lstlisting}

\begin{multiplechoice}
    \item 202 202 1618
    \item 202 404 1416
    \item 404 202 1618
    \item 404 808 810
\end{multiplechoice}

\label{q:hkoi:2022hj:q7}
\answer{ \ref{q:hkoi:2022hj:q7} 
A

x counts the sum of numbers divisible by 10 in range [1, 2022], in which the answer is
202.
y counts the sum of numbers divisible by 5 but not 10 in range [1, 2022], which is 202
as well.
z counts the rest of numbers, which is 2022-202-202=1618
}

\miquestion Write a program that keeps asking users to input a rating from 1 to 5, but will not terminate until the user enters 5. Different error messages should be shown when the user input is $\leq 0$, between 1 and 4, and $>5$. Assume the user only inputs integers.

Write two versions of the program, one with a do while loop, one without.

Expected input/ output:
\begin{lstlisting}
Give me a rating from one to five: -55
You can only input a positive number.
Give me a rating from one to five: 6
Thanks but you can at most give a rating of 5.
Give me a rating from one to five: 4
I think I deserve a better rating.
Give me a rating from one to five: 5 
Thank you.
//(and terminates)
\end{lstlisting}

\miquestion {\footnotesize(HKOI, UNI)} \textbf{(Prime Factorisation).} Write a function that takes a positive integer $\geq 2$ and print out all its prime factors separated by spaces.

Expected output:
\begin{lstlisting}
//On input 50
2 5 5
//On input 461244
2 2 3 7 17 17 19
\end{lstlisting}

\miquestion {\footnotesize(UNI)} You are required to design a program that translates a weather report from degree Fahrenheit to degree Celsius. There is always exactly 4 inputs, at 7,8,9,10am respectively.

\[C = \frac{{(F - 32) \times 5}}{{9}}\]

Expected input/ output:
\begin{lstlisting}
Input the temperature at 7am: 48.1
The temperature at 7am is 8.94444oC
Input the temperature at 8am: 50.2
The temperature at 8am is 10.1111oC
Input the temperature at 9am: 53.3
The temperature at 9am is 11.8333oC
Input the temperature at 10am: 54.8
The temperature at 10am is 12.6667oC
\end{lstlisting}

I would admit this question is a bit ambiguous and there is no one single correct answer as you are the one that makes the necessary design choices. The focus is to explore different ways we can utilise functions.

\begin{parts}
\part Write the program with the help of a function with the following signature that return the temperature of the argument in Celsius. 

\begin{lstlisting}
double f(double degf)
\end{lstlisting}

\part Write the same program again, but instead with the help of a function with the following signature that prints the temperature and the time on the screen.

\begin{lstlisting}
void g(int time, double degf)
\end{lstlisting}

\part Write the program again, but instead with the help of a function with the following signature that takes in the current time as an argument. It gets the temperature input from the user, and prints the temperature and time time on the screen.

\begin{lstlisting}
void h(int time)
\end{lstlisting}

\part Which implementation is the best in your opinion? Or is there another implementation that is better than all of the above? Discuss with your instructor.
\end{parts}

\subsection*{Loops and Arrays}
\miquestion {\footnotesize DSE\cite{dse:2016p1}} \code{NUM} is an integer array. What is the output of the following algorithm? 

\begin{lstlisting}
K = 0
while K <= 100 do
    NUM[K] = K * K
    K = K + 1
output ( NUM[3] + NUM[4] )
\end{lstlisting}

\begin{multiplechoice}
    \item 7
    \item 25
    \item 49
    \item 100
\end{multiplechoice}
\label{q:dse:2016p1:q32}

\answer{
    \ref{q:dse:2016p1:q32} B

    \if\aians1
    The algorithm you provided initializes an integer array NUM and populates it with the squares of numbers from 0 to 100. It then outputs the sum of NUM[3] and NUM[4].

Let's follow the algorithm step by step:

Initialize K as 0.\\
Enter the while loop.\\
Calculate NUM[K] as K * K and store the result in NUM[K].\\
Increment K by 1.\\
Repeat steps 3-4 until K becomes greater than 100.\\
Exit the while loop.\\
Output the sum of NUM[3] and NUM[4].\\

Given the algorithm, the output can be determined as follows:

The loop runs from K = 0 to K = 100 (inclusive).\\
On each iteration, the square of K is calculated and stored in NUM[K].\\
After the loop finishes, the values of NUM will contain the squares of numbers from 0 to 100.\\
Finally, the output is the sum of NUM[3] and NUM[4].\\

To find the specific output, we need to calculate NUM[3] and NUM[4] based on the algorithm:

NUM[3] is calculated as 3 * 3, which equals 9.\\
NUM[4] is calculated as 4 * 4, which equals 16.\\
Therefore, the output of the algorithm is the sum of NUM[3] and NUM[4], which is 9 + 16 = 25.
    \fi
}

\miquestion {\footnotesize HJ\cite{hkoi:2022hj}} What is the output of the following program?

\begin{lstlisting}
int dx[4] = {1, 2, 3, 4};
int dy[4] = {4, 3, 2, 1};
int x = 101, y = 100;
int i;
int main() {
    for(i = 1; i <= 100; i++) {
        x = x + dx[x % 4];
        y = y + dy[y % 4];
    }
    cout << x << ' ' << y;
    return 0;
}
\end{lstlisting}

\begin{multiplechoice}
    \item 1 4
    \item 499 500
    \item 495 500
    \item 351 350
\end{multiplechoice}

\label{q:hkoi:2022hj:q16}
\answer{ \ref{q:hkoi:2022hj:q16} 
B

Follow the trace table.
i: 1 2 3 4
x\%4: 1 3 3 3
y\%4: 0 0 0 0
x afterward: 103 107 111 115
y afterward: 104 108 112 116
Value of x and y can then be deduced, which is 103+99*4=499 and 100+100*4=500
respectively.
}

\miquestion {\footnotesize HJ\cite{hkoi:2022hj}} What is the output of the following program?

\begin{lstlisting}
int a[8] =
{2, 0, 2, 1, 1, 1, 1, 3};
int x;
int main () {
    x = 0;
    if (a[3] == 2)
        x = x + 1;
    else if (a[6] % 2 == 1)
        x = x + 2;
    if (a[4] == a[7])
        x = x + 3;
    else
        x = x + 4;
    cout << x;
    return 0;
}
\end{lstlisting}

\begin{multiplechoice}
    \item 1
    \item 2
    \item 4
    \item 6
\end{multiplechoice}

\label{q:hkoi:2022hj:q17}
\answer{ \ref{q:hkoi:2022hj:q17} 
D

a[3] = 2: false
a[6] mod 2 = 1: true
a[4] = a[7]: false
Thus x = 2 + 4 = 6
}

\miquestion Here is a generalisation to the code in \cref{sec:forloops}.

\begin{lstlisting}
int n = ...; //length of the array
int x[n] = ...; //the array
int sum = 0;
for(int i = 0; i < n; i++) { 
    sum += x[i];
}
cout << sum << endl; //31
\end{lstlisting}

\begin{parts}
\part Explain the purpose of the code.
\part A student suggested the following alternative code that is supposed to achieve the same purpose. Their instructor thinks it is incorrect. Explain why. Note that the code should work for any array \code{x} of any length with any values. 

\begin{lstlisting}
int n = ...; //length of the array
int x[n] = ...; //the array
int sum = x[0];
for(int i = 1; i < n; i++) {
    sum += x[i];
}
cout << sum << endl; 
\end{lstlisting}
\end{parts}

\label{q:sumx0}

\miquestion A student is asked to write a program to find the maximum element of an integer array. Here is their attempt:

\begin{lstlisting}
int n = ...; //length of the array
int x[n] = ...; //the array
int maximum = 0;
for(int i = 1; i < n; i++) {
    if(maximum <= x[i]) maximum = x[i];
}
cout << sum << endl; 
\end{lstlisting}

Again, their instructor thinks that it is incorrect. Explain why and write a correct version of the code. Note that the code should work for any array \code{x} of any length with any values. 

\miquestion Write a code snippet that finds the smallest and second smallest element of an array.

\begin{lstlisting}
int n = ...; //length of the array
int x[n] = ...; //the array
int minimum; //after your code finishes, minimum holds the smallest element
int minimum2nd; //after your code finishes, minimum2 holds the second smallest element
// your code goes here
\end{lstlisting}

\miquestion Rewrite both the code snippets in \cref{sec:loopbreak} and \cref{sec:loopcontinue} without using \code{break} and \code{continue}. 

\subsection*{Nested Loops}

\miquestion {\footnotesize HJ\cite{hkoi:2022hj}} Suppose a is an array of four 32 bit signed integers \code{int
a[4]}. Is the following two program segments always output equivalently? 

\begin{lstlisting}
for (w = 1; w <= a[0]; w++)
    for (x = 1; x <= a[1]; x++)
        for (y = 1; y <= a[2]; y++)
        for (z = 1; z <= a[3]; z++)
            cout << "HKOI" << endl;
\end{lstlisting}

\begin{lstlisting}
for (i = 0; i <= 3; i++)
    for (j = 1; j <= a[i]; j++)
        cout << "HKOI" << endl;
\end{lstlisting}

\label{q:hkoi:2022hj:q2}
\answer{ \ref{q:hkoi:2022hj:q2}
The first program prints ``HKOI'' for a[0]*a[1]*a[2]*a[3] times, but the second
does a[0]+a[1]+a[2]+a[3] times, which may not be equal. The statement is false.
}
\miquestion Create a function that takes in a non-negative number $n$ and prints the following pattern on the screen.

\begin{lstlisting}
1
1 2
1 2 3
.............
1 2 3 4 ... n
\end{lstlisting}

\miquestion {\footnotesize(HKOI, UNI)} \textbf{(Pascal Triangle).} Create a function that takes in a non-negative number $n$ and prints the first $n+1$ lines of the required pattern on the screen. We start counting column and row numbers from 0.

Expected output: $(n = 5)$
\begin{lstlisting}
1
1 1
1 2 1
1 3 3 1
1 4 6 4 1
1 5 10 10 5 1
\end{lstlisting}

The first and last entries of each line are 1. (i.e. $\Delta(j,0) = 1$ and $\Delta(j,j) = 1$ for all $j$)

The other $(i,j)$th entries are given by the sum of $(i-1,j-1)$th and $(i-1,j)$th entries. (i.e. $\Delta(i,j) = \Delta(i-1,j-1) + \Delta(i-1,j)$ for all $i,j$ with $0 < j < i$)

How much memory is used? Can we do better? Discuss with your tutor.
\end{questions}
% Translated to Java

\if\proglang1
\chapter{Java Knowledge}
\else
\chapter{C++ Knowledge}
\fi


Here are some basic\if\proglang1 Java \else ~C++ \fi knowledge, I bet you have seen some of the concepts in this chapter already on other occasions.

% Translated to Java end 

\section{Conditionals}
\subsection{\texttt{if} statements}

\begin{lstlisting}
int score;
cin >> score;
if(score >= 70){
    printf("Good job.\n");
}else if(score >= 40){
    printf("You got a pass.\n");
}else{
    printf("You failed.\n");
}
\end{lstlisting}

\subsection{\texttt{switch case}}

Just looks neater when you are testing on the same variable multiple times. You can of course use if else if else if... 

Note that it only works for int and char, and only equality tests are allowed. For example, the above example on scores could not be replaced using switch case.

Also note that the \textbf{\texttt{break}} keyword is necessary to quit the switch statement, or else it will run the default clause after the \texttt{case} clause.

\begin{lstlisting}
char x;
cin >> x;
switch(x){
    case 'z':
        cout << "It is the last letter of the alphabet." << endl;
        break;
    //this is how you do multiple equality tests
    case 'a':
    case 'e':
    case 'i':
    case 'o':
    case 'u':
        cout << "It is a vowel." << endl;
        break;
    //equivalent to the else clause
    default:
        cout << "It is not a vowel." << endl;
        break;
}
\end{lstlisting}

% Translated to Java

\section{Arrays}
\label{sec:arrayintro}
Arrays store sequences of data of the same type.

\if\proglang1
\begin{lstlisting}
int[] x = {3,1,4,1,5,9,2,6};

System.out.println(x[0]); //3 
System.out.println(x[4]); //5
System.out.println(x[7]); //6 (last element)
System.out.println(x[8]); //Error 
\end{lstlisting}
\else
\begin{lstlisting}
int x[8] = {3,1,4,1,5,9,2,6};
//alternatively: int x[] = {3,1,4,1,5,9,2,6}; length of array can be omitted as the compiler can derive it from the right hand side

cout << x[0] << endl; //3 
cout << x[4] << endl; //5
cout << x[7] << endl; //6 (last element)
cout << x[8] << endl; //unexpected value (Why?)
\end{lstlisting}
\fi

We call the things stored in the array \textbf{elements}. We distinguish different elements using \textbf{index}. The index \textbf{starts from 0}, and ends at $n-1$, where $n$ is the length of the array. The \textbf{length} of the array refers to the maximum number of elements that the array can store, based on the memory allocated when it is \textbf{initialised}. We cannot change the length of the array normally.

\includegraphics[width=12cm]{images/ch2-arrayindex.png}
\vspace{6mm}

There are two ways of initialising arrays. The first way is to hard-code the elements in the array. The length of the array is inferred by the number of elements given, and it cannot be extended afterwards.

\if\proglang1
\begin{lstlisting}
int[] x = {3,1,4,1,5,9,2,6};
\end{lstlisting}
\else
\begin{lstlisting}
int x[] = {3,1,4,1,5,9,2,6};
\end{lstlisting}
\fi

The second way is when you are not quite sure what the elements of the array are yet, but you must still supply the length of the array.

\if\proglang1
\begin{lstlisting}
int[] x = new int[8];
\end{lstlisting}
\else
\begin{lstlisting}
int x[8];
\end{lstlisting}
\fi

% Translated to Java end

\section{Loops}
\subsection{\texttt{for} loops}
Runs the body a specified number of times.

\begin{lstlisting}
int x[8] = {3,1,4,1,5,9,2,6};
int sum = 0;
for(int i = 0; i < 8; i++) { //loops i=0,1,2,3,4,5,6,7
    sum += x[i];
}
cout << sum << endl; //31
\end{lstlisting}

\begin{lstlisting}
int x[8] = {3,1,4,1,5,9,2,6};
int sum = 0;
for(int i = 7; i >= 0; i--) { //loops i=7,6,5,4,3,2,1,0
    sum += x[i];
}
cout << sum << endl; //31
\end{lstlisting}

\subsection{\texttt{while} loops}
Runs the body until the test is false.
\begin{lstlisting}
int x[8] = {3,1,4,1,5,9,2,6};
int sum = 0;
int i = 0;
while(i<8) { 
    sum += x[i]; //loops i=0,1,2,3,4,5,6,7
    i++;
}
cout << sum << endl; //31
\end{lstlisting}

\begin{lstlisting}
int x[8] = {3,1,4,1,5,9,2,6};
int sum = 0;
int i = 0;
while(i<8&&sum<10) { 
    sum += x[i];
    i++;
}
cout << sum << endl; //14
\end{lstlisting}

\subsection{\texttt{do while} loops}
Runs the body until the test is false. The body will run at least once.

\begin{lstlisting}
bool emergency = false;
do{
    printf("EMERGENCY\n"); //will be printed
}while(emergency);
\end{lstlisting}

\begin{lstlisting}
bool emergency = false;
while(emergency){
    printf("EMERGENCY\n"); //will not be printed
}
\end{lstlisting}

This example yields the same result with while loops because the test is true initially, allowing the loop to run at least once.

\begin{lstlisting}
int x[8] = {3,1,4,1,5,9,2,6};
int sum = 0;
int i = 0;
do { 
    sum += x[i]; //loops i=0,1,2,3,4,5,6,7
    i++;
}while(i<8);
cout << sum << endl; //31
\end{lstlisting}

\subsection{Infinite loops}

If the test case is always true (there is nothing in the loop to make the test false). The loop is an infinite loop\index{infinite loop}, where the loop will not terminate (until resources are used up), and the remaining parts of the program can never be run. In this example, EMERGENCY will be printed forever nonstop.

\begin{lstlisting}
bool emergency = true;
while(emergency){
    printf("EMERGENCY\n");
}
destroy_world(); //will not be run
\end{lstlisting}
\vspace{6mm}

\subsection{\texttt{break}}

\texttt{break} allows you to terminate the loop early.

\begin{lstlisting}
int x[8] = {3,1,4,1,5,9,2,6};
int sum = 0;
int i = 0;
while(i<8) { 
    sum += x[i];
    i++;
    if(sum >= 10) break;
}
cout << sum << endl; //14
\end{lstlisting}

\subsection{\texttt{continue}}

\texttt{continue} allows you to jump to the next iteration, skipping the rest of the current iteration early.

\begin{lstlisting}
int x[8] = {3,1,4,1,5,9,2,6};
int sum = 0;
int i = 8;
while(i>0) {
    i--;
    if(x[i] >= 5) continue;
    sum += x[i];
}
cout << sum << endl; //11
\end{lstlisting}

\subsection{Necessity of \texttt{break} and \texttt{continue}}

% Note that all for loops and do while loops can be written as a while loop.

% \begin{lstlisting}[basicstyle=\rmfamily]
% for(int i = 0; i < <target>; i++) <body> === 
% int i = 0; while(i < <target>) <body> i++;
% \end{lstlisting}

% \begin{lstlisting}[basicstyle=\rmfamily]
% do <body> while(<test>); === <body> while(<test>) <body>
% \end{lstlisting}

Note that the use of \texttt{break} and \texttt{continue} can be avoided by modifying the loop condition and using if statements respectively. 

In fact, plenty of other programming languages do not have \texttt{break} and \texttt{continue}. 

\section{2D arrays}

Don't have time to cover, there should be plenty of resources online on this topic.

\begin{lstlisting}
int x[3][] = {{1,4,7},{2,5,8},{3,6,9}};
cout << x[2][1] << endl; //6
\end{lstlisting}

\subsection{Nested loops}

\begin{lstlisting}
int x[3][] = {{1,4,7},{2,5,8},{3,6,9}};
sum = 0;
for(int i=0;i<3;i++)
    for(int j=0;j<3;j++)
        sum += x[i][j];
cout << sum << endl; //45
\end{lstlisting}

% Translated to Java 
\section{\if\proglang1 Methods \else ~Functions \fi}
\label{sec:functions}

\if\proglang1 Methods \else Functions \fi let you organize code better and reduce the amount of repeated code. You can define them by:

\if\proglang1
\begin{lstlisting}[language=,basicstyle=\rmfamily]
public static <return type> <name>(<arguments>){
    return <return value>;
}
\end{lstlisting}
\else
\begin{lstlisting}[language=,basicstyle=\rmfamily]
<return type> function(<arguments>){
    return <return value>;
}
\end{lstlisting}
\fi

For example, this \texttt{fact}\if\proglang1 method \else ~function \fi accepts an integer x, and returns another integer.
\if\proglang1
\begin{lstlisting}
public static int fact(int x){
    int y = 1;
    for(int i = 1; i <= x; i++){
        y *= i;
    }
    return y;
}
\end{lstlisting}
\else
\begin{lstlisting}
int fact(int x){
    int y = 1;
    for(int i = 1; i <= x; i++){
        y *= i;
    }
    return y;
}
//What would happen if we input a negative number?
\end{lstlisting}
\fi

This is how you use a\if\proglang1 method \else ~function \fi: 
You could call it by just supplying the argument.

\if\proglang1
\begin{lstlisting}
    System.out.println(fact(6)); //720
\end{lstlisting}
\else
\begin{lstlisting}
    cout << fact(6) << endl; //720
\end{lstlisting}
\fi

Alternatively, we usually would prefer saving the value returned for later use.

\begin{lstlisting}
    int x = fact(6); //720 is now stored in x
\end{lstlisting}

You could use the \texttt{\textbf{void}} keyword to indicate that a\if\proglang1 method \else ~function \fi does not have a return value. For \texttt{void}\if\proglang1 methods \else ~functions \fi, if there is no explicit \texttt{return;} statement, the\if\proglang1 method \else ~function \fi will automatically quit at the end of the\if\proglang1 method \else ~function \fi. 

\if\proglang1
\begin{lstlisting}
import java.util.Scanner;

public static void giveComment(int score){
    if(score >= 70){
        System.out.println("Good job.");
    }else if(score >= 40){
        System.out.println("You got a pass.");
    }else{
        System.out.println("You failed.");
    }
    //automatically quits here
}

public static void main(String args[]){
    Scanner scanner = new Scanner(System.in);  
    int s = scanner.nextInt();
    giveComment(s);
}
\end{lstlisting}
\else
\begin{lstlisting}
void giveComment(int score){
    if(score >= 70){
        printf("Good job.\n");
    }else if(score >= 40){
        printf("You got a pass.\n");
    }else{
        printf("You failed.\n");
    }
    //automatically quits here
}

int main(){
    int s;
    cin >> s;
    giveComment(s);
    //returning 0 is optional for some C/C++ compilers, it is implied it returns 0 normally
}
\end{lstlisting}
\fi

The main\if\proglang1 method \else ~function \fi is just a\if\proglang1 method \else ~function \fi with no arguments and an integer as a return value. The main\if\proglang1 method \else ~function \fi (specially recognized by the compiler) is the point of entry of the program, the program terminates when the main program returns, returning 0 indicates that there is no error, and returning other integers indicate otherwise.
\vspace{6mm}

Note that the giveComment\if\proglang1 method \else ~function \fi must be placed in front of the main\if\proglang1 method \else ~function \fi, or else the main\if\proglang1 method \else ~function \fi could not call the giveComment\if\proglang1 method \else ~function \fi as the\if\proglang1 method \else ~function \fi has not been defined by the time the compiler reads till the main\if\proglang1 method \else ~function \fi. 

% Translated to Java end
\section*{Exercises}
\addcontentsline{toc}{section}{Exercises}

\begin{questions}

\miquestion What is the output of the following program? \footnotecite{hkoi:2022hs}
\begin{lstlisting}
int n, i, x;
string s = "hkoi.org";
int main() {
    n = strlen(s);
    x = 0;
    for (i = 0; i < n; i++)
        if (s[i] != s[n - i - 1]) {
        s[i] = s[n - i - 1];
        x++;
    }
    cout << x;
    return 0;
}
\end{lstlisting}

\label{q:hkoi:2022hs:q2}

\answer{
    \ref{q:hkoi:2022hs:q2} 3\\\\
    It suffices to trace the program directly. The program counts the edit distance between the string and the string reversed, divided by 2.
}

\miquestion Which of the following Boolean expressions is not logically equivalent to other Boolean expressions? \footnotecite{hkoi:2022hs}
\begin{multiplechoice}
\item \code{A OR (NOT B OR NOT C)}
\item \code{NOT (((NOT A) AND B) AND C)}
\item \code{(A OR NOT B) OR NOT C}
\item \code{((NOT A) AND B) OR NOT C}
\end{multiplechoice}

\label{q:hkoi:2022hs:q5}

\answer{
    \ref{q:hkoi:2022hs:q5} D\\\\
    NOT (((NOT A) AND B) AND C)\\
    $\equiv$ NOT((NOT A) AND B) OR NOT C (by De Morgan's Law)\\
    $\equiv$ (A OR NOT B) OR NOT C (by De Morgan's Law)\\
    $\equiv$ A OR (NOT B OR NOT C) (by commutative law)\\
    When A $\equiv$ TRUE, B $\equiv$ TRUE, C $\equiv$ FALSE, option D gives FALSE but option A, B and C gives TRUE.
}


% TODO: HKOI 2022 SENIOR Q13

% TODO: 2022 JUNIOR Q5, 10, 11, order of precedence
\end{questions}
\chapter{Data types}

C++ is a strongly typed language, i.e. you have to specify the type of variables when initializing them, and variables could not store data of different types.  

\section{Primary data types}
Primary data types are those that are not composed of other data types.

Basic knowledge: 1 bit is a single 0 or 1 in memory, 8 bits form 1 byte, we usually discuss memory in terms of bytes instead of bits.

\begin{table}[h]
    \centering
    \begin{tabular}{|m{6em}|m{6em}|m{10em}|m{12em}|}
        \hline
        \textbf{Data Type} & 
        Bytes in Memory & 
        Range & 
        Remarks 
        \\ \hline \hline
        
        \texttt{int} &
        4 & 
        $-2^{31}$ to $2^{31}-1$ (i.e. -2147483648 to 2147483647) &
        
        \\ \hline
        
        \texttt{bool} &
        1 & 
        true / false  &
        \tablefootnote{Out of scope: in C++ bool is a primary data type, in C you will have to include stdbool.h to use it.} 
        \\ \hline
        
        \texttt{char} &
        1 & 
        /  &
        Stored using an integer, the ASCII code of the character involved.
        \\ \hline
        
        \texttt{float} &
        4 &
        A large range \textcolor{gray}{ (~$-10^{38}$ to $10^{38}$)} &
        Stores numbers with decimal places
        \\ \hline
        
        \texttt{double} &
        8 & 
        A large range \textcolor{gray}{ (~$-10^{308}$ to $10^{308}$)} &
        Stores numbers with decimal places
        \\ \hline
    \end{tabular}
\end{table}
\pagebreak

These are variants of integers. They are of less importance.\footnote{Reference: \href{https://en.cppreference.com/w/cpp/language/types}{https://en.cppreference.com/w/cpp/language/types}, the range and bytes occupied differs for different version of C++}

\begin{table}[h]
    \centering
    \begin{tabular}{|m{6em}|m{6em}|m{10em}|m{12em}|}
        \hline
        \textbf{Data Type} & 
        Bytes in Memory & 
        Range & 
        Remarks 
        \\ \hline \hline
        
        \texttt{unsigned} &
        4 & 
        $0$ to $2^{32}-1$ (i.e. 0 to 4294967295) &
        Cannot store negative numbers, yet approximately double the range.
        \\ \hline
        
        \texttt{short} &
        2 & 
        $-2^{15}$ to $2^{15}-1$ (i.e. -32768 to 32767) &
        Half the size of an \texttt{int}
        \\ \hline
        
        \texttt{long long} &
        8 & 
        $-2^{63}$ to $2^{63}-1$ &
        Double the size of an \texttt{int}
        \\ \hline
        
    \end{tabular}
\end{table}

\section{Integer \index{overflow}}

As you saw in the last section, integers have a limited range. Overflow occurs when the value we want to store is out of the range that the data type can represent. 

For integers (\texttt{int},\texttt{unsigned},\texttt{short},\texttt{long long}), the value it represents will go to the other end of the range when they overflow.\footnote{Reason behind out of scope: requires understanding of two's compliment}. 

For example,
\begin{lstlisting}
int x = 2147483647;
cout << "Before overflow: " << x << endl; //2147483647
x += 1;
cout << "After overflow: " << x << endl; //-2147483648

unsigned y = 4294967295;
cout << "Before overflow: " << y << endl; //4294967295
y += 1;
cout << "After overflow: " << y << endl; //0
\end{lstlisting}

\section{Floating point values}

OK, the range of floating point values (\texttt{float} and \texttt{double}) are much larger than ints, so is it a good idea to use them all the time? NO!

Floating point values store values inaccurately, and rounding errors accumulate. Even though using \texttt{double} reduces the rounding error, floating point values are still nasty in some ways. 

You are advised to only use \texttt{double} to store numbers with decimal points and very large numbers (those can't be represented by \texttt{long long}). And never use \texttt{float}\footnote{Memory usage is not a problem nowadays}.

\section{Boolean operators}

A \index{Boolean} variable (\texttt{bool}) can be either \texttt{true} or \texttt{false}. There are three operators that we can perform on \texttt{bool}s. Including AND (\&\&), OR (\textbar\textbar), NOT (!).

\begin{table}[h]
    \centering
    \begin{tabular}{|m{4em}|m{4em}|m{4em}|m{4em}|m{4em}|}
        \hline
        a & 
        b & 
        a \&\& b & 
        a\textbar\textbar b & 
        !a 
        \\ \hline \hline
        
        true & 
        true & 
        true & 
        true & 
        false 
        \\ \hline
        
        true & 
        false & 
        false & 
        true & 
        false 
        \\ \hline
        
        false & 
        true & 
        false & 
        true & 
        true 
        \\ \hline
        
        false & 
        false & 
        false & 
        false & 
        true 
        \\ \hline
        
    \end{tabular}
\end{table}

C/C++ recognizes 0 as false and other non-zero values as true. (using 1 is usually the convention) It is common to use 0 instead of false and 1 instead of true.

You could do \texttt{while(true) ...} for an infinite loop, but you might as well do \texttt{while(1) ...}
\vspace{6mm}

Order of precedence: ! $>$ \&\& $>$ \textbar\textbar

For example, 

!0 \textbar\textbar 0 \&\& 1

= (!0) \textbar\textbar 0 \&\& 1

= 1 \textbar\textbar 0 \&\& 1

= 1 \textbar\textbar (0 \&\& 1)

= 1 \textbar\textbar 0

= 1

\subsection{Bitwise operators}

Don't have time to cover, there should be plenty of resources online on this topic. 

Yet, it is important for senior group contestants as this kind of questions seem common. 

\includegraphics[width=7cm]{ch3-bitwise.png}
\footnote{Reference: \href{https://www.digitalocean.com/community/tutorials/python-bitwise-operators}{https://www.digitalocean.com/community/tutorials/python-bitwise-operators}}

\section{Characters are integers}

Let's run some C++ code.

\begin{lstlisting}
char a = 'a'; //we use single quotes for characters and double quotes for strings
printf("%d %c\n",a,a); //97 a
a = 97;
printf("%d %c\n",a,a); //97 a
a = 65;
printf("%d %c\n",a,a); //65 A
a += 1;
printf("%d %c\n",a,a); //66 B
\end{lstlisting}

It seems that each character is associated with an integer. That's right! 128 (later extended to 256) frequently used characters are chosen and they are associated with an integer between 0 and 127.\footnote{ASCII Table: \href{https://simple.wikipedia.org/wiki/ASCII}{https://simple.wikipedia.org/wiki/ASCII}}

We can perform arithmetic operations with \texttt{char} variables, as they are just integers! 

\section{Strings}

Strings are just an array of characters, but there are a few nasty catches.

\subsection*{Strings terminate with a null character \texttt{\textbackslash 0}}

Let's think about what \texttt{char s[10] = "Hello";} does. It is actually equivalent to 

\includegraphics[width=16cm]{ch3-nullstring.png}

The \texttt{\textbackslash 0} (ASCII code = 0) indicates the end of a string, it is necessary for C/C++ programs so that they could identify when the string ends. If you remove it by hand something bad would happen...

Don't trust me? Try running:

\begin{lstlisting}
char s[4] = "FGH"; //without the terminating character
s[3] = 'I';
cout << s << endl; //FGHI followed by a bunch of nonsense, the nonsense is different everytime we run the program

char s1[7] = { 'A', 'B', 'C', '\0', 'D', 'E', '\0'};
cout << s1 << endl; //ABC
\end{lstlisting}

\includegraphics[width=3cm]{ch3-beyondnull.png}

The first example shows what happens when we explicitly remove the null character at the end of the string. (By replacing \texttt{FGH\textbackslash 0} with \text{FGHI}) It attempts to access and print things beyond the memory allocated. (see chapter 5 for more details)

The second example verifies that the null character is recognized by C/C++ as the end of the string, despite having some other contents after the null character.

\textbf{So we always have to reserve one more space for the null character.}

\subsection*{\texttt{cin} obtains input word by word by default}

\begin{lstlisting}
char s2[20];
cin >> s2; //You input Hello world
cout << s2 << endl; //Hello
\end{lstlisting}

This example shows that \texttt{cin} only catches the first word and store it in s2, while leaving other words for another variable.

\begin{lstlisting}
char s2[20], s3[20];
cin >> s2; //You input Hello world
cout << s2 << endl; //Hello

cin >> s3; //The computer did not ask for your input
cout << s3 << endl; //world
\end{lstlisting}

However, sometimes you want to obtain a whole line of text input rather than just a word. You could use \texttt{cin.getline}. It accepts two arguments, first the string variable (the character array you defined), second the maximum amount of characters it should get (you have to set it equal to the size of the string)

\begin{lstlisting}
char s4[20];
cin.getline(s4,20); //You input Good morning
cout << s4 << endl; //Good morning
\end{lstlisting}

Unfortunately it is not as simple as that, things get complicated when you have other \texttt{cin}s in the same program.

\begin{lstlisting}
char s2[20], s3[20];
cin >> s2; //You input Hello world
cout << s2 << endl; //Hello

cin >> s3; //The computer did not ask for your input
cout << s3 << endl; //world

char s4[20];
cin.getline(s4,20); //The computer did not ask for your input
cout << s4 << endl; //Probably a next line symbol
\end{lstlisting}

This behaviour is probably\footnote{Who is certain about this chaos?} due to the next line symbol of the Hello world line has not been read, and \texttt{cin.getline} reads the text until it hits a next line symbol, so the only thing stored in s4 is the next line symbol...

What you could do is use \texttt{cin.clear} and \texttt{cin.ignore(10000,'\textbackslash n'} before \texttt{cin.getline}, in order to clear the buffer.

\begin{lstlisting}
char s2[20], s3[20];
cin >> s2; //You input Hello world
cout << s2 << endl; //Hello

cin >> s3; //The computer did not ask for your input
cout << s3 << endl; //world

char s4[20];
cin.clear();
cin.ignore(10000, '\n');
cin.getline(s4,20); //You input Good morning
cout << s4 << endl; //Good morning
\end{lstlisting}

Alternatively, you could call \texttt{cin.getline} twice, the first one to read the previous line, and the second one to actually read your input. The result of the first read is overwritten by the second one as desired.

\begin{lstlisting}
char s2[20], s3[20];
cin >> s2; //You input Hello world
cout << s2 << endl; //Hello

cin >> s3; //The computer did not ask for your input
cout << s3 << endl; //world

char s4[20];
cin.getline(s4,20); //read the previous line
cin.getline(s4,20); //You input Good morning
cout << s4 << endl; //Good morning
\end{lstlisting}

\subsection{Strings VS string variables}

The following two restrictions concern string variables but not strings. Now let's differentiate strings and string variables.

\subsection*{String variables cannot be compared normally}

Strings are only for one time use. You can compare them normally using comparison operators.

\begin{lstlisting}
cout << ("ABC"=="DEF") << endl; //0
cout << ("ABC" < "DEF") << endl; //1 (why?)
cout << ("abc" <= "ABC") << endl; //0 (why?)
\end{lstlisting}

String variables are character arrays that are used to store strings. Sadly they cannot be compared normally. (use strcmp instead)\footnote{Reason out of scope: because the references of the arrays are compared instead of their values. You may understand more in Chapter 5 but the full concept is not necessary}

\begin{lstlisting}
char s5[20] = "ABC", s6[20] = "ABC", s7[20] = "DEF";
cout << (s5==s5) << endl; //1
cout << (s5==s6) << endl; //0
cout << (s5<s6) << endl; //0
\end{lstlisting}

\subsection*{The whole string variable cannot be reassigned}

Thought you could do this? Nah...

\begin{lstlisting}
char s8[20] = "ABCD";
s8 = "XYZ"; //SYNTAX ERROR
\end{lstlisting}

Well you could do the following instead if you are desperate, but there is a better way in the next section. (strcpy)

\begin{lstlisting}
s8[0] = 'X'; 
s8[1] = 'Y'; 
s8[2] = 'Z'; 
s8[3] = '\0'; 
\end{lstlisting}


\subsection{Operation on string variables (IMPORTANT)}

\begin{table}[h]
    \centering
    \begin{tabular}{|m{10em}|m{25em}|}
        \hline
        \textbf{Function} & 
        Usage 
        \\ \hline \hline
        
        \texttt{strlen(s)} &
        Get string length
        \\ \hline
        
        \texttt{strcpy(s1, s2)} &
        Copy value stored in \texttt{s2} to \texttt{s1} 
        \\ \hline
        
        \texttt{strcat(s1, s2)} &
        Concatenate (i.e. combine) the two strings, store it in \texttt{s1} 
        \\ \hline
        
        \texttt{strcmp(s1, s2)} &
        Compare two strings, returns 0 if \texttt{s1} = \texttt{s2}, returns a negative number if \texttt{s1} $<$ \texttt{s2}, returns a positive number if \texttt{s1} $>$ \texttt{s2}. (In dictionary order with respect to ASCII code)
        \\ \hline
    \end{tabular}
\end{table}

You are advised to try out these operations yourself, I am sure you will find surprises.

\section{Conclusion and further resources}

\href{https://www.youtube.com/watch?v=Zy2bNkSxv8M}{This video}\footnote{Link: \href{https://www.youtube.com/watch?v=Zy2bNkSxv8M}{https://www.youtube.com/watch?v=Zy2bNkSxv8M}} summarizes some more nasty things about strings. Bear in mind what you can and what you can't do with them.
\vspace{6mm}

\begin{center}
\textit{The most comfortable type to deal with are integers.}
\end{center}

% \miquestion Suppose you have two integer variables \code{a} and \code{b}, write a code snippet that swaps the value of \code{a} and \code{b} if necessary so that \code{a} is always the larger of the two.

% \begin{lstlisting}
% int a,b;
% cin >> a >> b; 
% // your code goes here
% cout << "the larger element is " << a << " while the smaller is " << b << endl;
% \end{lstlisting}

% For example, if \code{a=4}, \code{b=5} before your code, it should be \code{a=5}, \code{b=4} after. If \code{a=8}, \code{b=5} before your code, it should be \code{a=8}, \code{b=5} after (not swapped).

% Tell them to write a function instead
\chapter{Recursion}

Recursive functions are those which call themselves, for example, this implementation of the factorial function.

It might be difficult to understand at first, but it simplifies code once you get used to it. The best way to master it is through looking at more examples. You can see more examples on recursion in the mycodeschool playlist.

\section{Further resources: mycodeschool}

Playlists by 
\href{https://www.youtube.com/user/mycodeschool}{mycodeschool}\footnote{Link: \href{https://www.youtube.com/user/mycodeschool}{https://www.youtube.com/user/mycodeschool}}
that focus on 
\href{https://www.youtube.com/watch?v=_OmRGjbyzno&list=PL2_aWCzGMAwLz3g66WrxFGSXvSsvyfzCO}{recursion}\footnote{Link: \href{https://www.youtube.com/watch?v=_OmRGjbyzno&list=PL2_aWCzGMAwLz3g66WrxFGSXvSsvyfzCO}{https://www.youtube.com/\\watch?v=\_OmRGjbyzno\&list=PL2\_aWCzGMAwLz3g66WrxFGSXvSsvyfzCO}}, 
\href{https://youtube.com/playlist?list=PL2_aWCzGMAwLZp6LMUKI3cc7pgGsasm2_}{pointers}\footnote{Link: \href{https://youtube.com/playlist?list=PL2_aWCzGMAwLZp6LMUKI3cc7pgGsasm2_}{https://youtube.com/playlist?list=PL2\_aWCzGMAwLZp6LMUKI3cc7pgGsasm2\_}} and 
\href{https://www.youtube.com/playlist?list=PL2_aWCzGMAwI3W_JlcBbtYTwiQSsOTa6P}{data structures (linked lists, stacks, queues)}\footnote{Link: \href{https://www.youtube.com/playlist?list=PL2_aWCzGMAwI3W_JlcBbtYTwiQSsOTa6P}{https://www.youtube.com/playlist?list=PL2\_aWCzGMAwI3W\_JlcBbtYTwiQSsOTa6P}}.
mycodeschool explains concepts about computing pretty well. You can watch his other playlists if interested.

\section{Example: Factorial function}

\begin{lstlisting}
int fact(int x){
    if(x==0) return 0;
    else return x*fact(x-1);
}
//What would happen if we input a negative number?
\end{lstlisting}

Internally, there is a \textbf{call stack}\index{call stack} to store information about each function call (e.g. the values of the variables, what is left to do) so that they could resume in order.

The figure shows what happens when I call fact(4)

\includegraphics[width=15cm]{ch4-factorial.png}



\include{chapter4ex}
\chapter{Arrays and Linked Lists}

We will talk more about arrays in this chapter, with new concepts - linked lists, pointers and time complexity.

\section{Arrays in memory}

Let's go back to the example in Chapter 2.

\begin{lstlisting}
int x[8] = {3,1,4,1,5,9,2,6};
//alternatively: int x[] = {3,1,4,1,5,9,2,6}; length of array can be omitted as it can be derived from the right hand side

cout << x[0] << endl; //3 
cout << x[4] << endl; //5
cout << x[7] << endl; //6 (last element)
cout << x[8] << endl; //unexpected value 
\end{lstlisting}

The reason for x[8] to return unexpected value rather than an error is pretty interesting. To understand that, you will need to know how arrays are stored, and how array elements are accessed. 

A \textbf{continuous} memory is allocated for each array. As indicated in the figure.

\includegraphics[width=13cm]{ch5-array.png}

C++ stores the base address of the array x, say at location 2240. Then to obtain x[k], it will calculate the address that it needs to retrieve by 2240 + k*4, the *4 is there because each integer occupies 4 bytes of memory, the multiplier is different depends on different datatypes. 

For example, when querying x[4], we obtained it from address 2240+16 = 2256. 

So, when querying x[8], we obtained it from address 2240+32 = 2272. What is in address 2272? We don't know! It is just a bunch of random 0s and 1s last modified by other programs. 
\vspace{6mm}

If you run this program multiple times, the result printed is different, this is because the base address of the array, allocated by the operating system and the C++ compiler, is different every time. 

Let's say, next time it ended up at base address 8084, then it will obtain x[8] at 8116, it probably stores a different sequence of 0s and 1s. While x[0..7] are the same, because that is what \texttt{int x[8] = {3,1,4,1,5,9,2,6};} does, put 3 at base address, put 1 at base address + 4, put 4 at base address + 8 ..., once the base address is allocated.

\section{Pointers}

Don't have time to cover, there should be plenty of resources online on this topic, including mycodeschool's playlist on pointers.

\section{Call by reference and call by value}

Testing testing

\section{Linked lists}

Scattered blocks of memory are allocated for the linked list. Each block of memory is called a \textbf{node}\index{node}. It contains a datum and also the reference to the next node. The only information we have is the reference to the first node, the \textbf{head}\index{head}. We will be able to read the whole list by \textbf{traversing}\index{traversing} the linked list, that is, to read the datum of the node and then proceed to another node by following the reference. Eventually if we reach a node with next = null, indicating that the end of the list is reached. As indicated in the figure.

\includegraphics[width=15cm]{ch5-linkedlist1}

\section{Contrasting linked lists and arrays}

\includegraphics[width=15cm]{ch5-linkedlist2}

\includegraphics[width=15cm]{ch5-linkedlist3}

\section{Time complexity}



\section{Conclusion}
As you can see, linked lists and arrays have different efficiencies when performing different operations. There is no 'better' data structure, it is the job of we computer scientists to figure out which data structure should be used. Both linked lists and arrays can be used to implement stacks and queues and the algorithms that I will mention. 

Yet I do not advise you to code using linked lists in C++, because it is quite troublesome without automatic garbage collecting and with an inconvenient object oriented programming infrastructure. It is best done using other languages like Java, Scala, or Python.


\include{chapter5ex}
\chapter{Stacks and Queues}

Stacks and queues are data structures, adapted from our daily life. We realized that we can implement stacks and queues using arrays or linked lists. We will look into them in this chapter. We will look at the concept on how to implement them efficiently and write C++ code for the array implementation.

\includegraphics[width=16cm]{images/ch6-stackqueue.png}

\section{Further resources: mycodeschool}

Playlists by 
\href{https://www.youtube.com/user/mycodeschool}{mycodeschool}\footnote{Link: \href{https://www.youtube.com/user/mycodeschool}{https://www.youtube.com/user/mycodeschool}}
that focuses on 
\href{https://www.youtube.com/watch?v=_OmRGjbyzno&list=PL2_aWCzGMAwLz3g66WrxFGSXvSsvyfzCO}{recursion}\footnote{Link: \href{https://www.youtube.com/watch?v=_OmRGjbyzno&list=PL2_aWCzGMAwLz3g66WrxFGSXvSsvyfzCO}{https://www.youtube.com/\\watch?v=\_OmRGjbyzno\&list=PL2\_aWCzGMAwLz3g66WrxFGSXvSsvyfzCO}}, 
\href{https://youtube.com/playlist?list=PL2_aWCzGMAwLZp6LMUKI3cc7pgGsasm2_}{pointers}\footnote{Link: \href{https://youtube.com/playlist?list=PL2_aWCzGMAwLZp6LMUKI3cc7pgGsasm2_}{https://youtube.com/playlist?list=PL2\_aWCzGMAwLZp6LMUKI3cc7pgGsasm2\_}} and 
\href{https://www.youtube.com/playlist?list=PL2_aWCzGMAwI3W_JlcBbtYTwiQSsOTa6P}{data structures (linked lists, stacks, queues)}\footnote{Link: \href{https://www.youtube.com/playlist?list=PL2_aWCzGMAwI3W_JlcBbtYTwiQSsOTa6P}{https://www.youtube.com/playlist?list=PL2\_aWCzGMAwI3W\_JlcBbtYTwiQSsOTa6P}}.
mycodeschool explains concepts about computing pretty well. You can watch his other playlists if interested.

\section{Stack}

\includegraphics[width=16cm]{images/ch6-stackarray.png}

\includegraphics[width=16cm]{images/ch6-stacklinkedlist.png}
\section{Queue}

% TODO: HKOI 2022 SENIOR Q14(?)
\chapter{Searching Algorithms}

In this chapter we will look into some algorithms that search for a specific integer\footnote{The algorithms also work for other data types. However integer is selected for easier understanding.} in an array, and return its index in the array. We will look at how they work and analyze their efficiency (using time complexity). We will only focus on implementation using arrays.

\section{Further resources (Ch 7-8)}

Watching animations of these algorithms is useful in understanding how they work. I don't have a specific YouTube channel in mind. Yet algorithms in this and the next chapter are all very famous, I am sure you can find plenty of them online.

\section{Linear search}

Linear search returns the location of the leftmost occurrence of the integer when there are duplicates, and returns \textit{n} if the integer is not found.

\begin{lstlisting}
int main(){
    int x[] = {3,1,4,1,5,9,2,6};

    cout << linearSearch(x,8,1) << endl; //1 (leftmost occurance)
    cout << linearSearch(x,8,2) << endl; //6
    cout << linearSearch(x,8,7) << endl; //8 (not found, return n)
}
\end{lstlisting}

It might not be completely clear whether the target is in the array or not, we have to check by \texttt{i$<$n}:

\begin{lstlisting}
int i = linearSearch(x,8,target);
if(i<n)
    cout << target << " found at location " << i << endl;
else 
    cout << target << " not found" << endl;
\end{lstlisting}

Now, after we have clarified what it does, let's implement it in C++.
\vspace{6mm}

C++: (\textit{Exercise: Try to re-implement yourself.})
\begin{lstlisting}
int linearSearch(int x[], int n, int target){
    int i = 0;
    while(i<n){
        if(x[i]==target) break;
        i++;
    }
    return i;
}
\end{lstlisting}

If you want to eliminate the \texttt{break} statement, you could replace the loop by:

\begin{lstlisting}
while(i<n&&x[i]!=target) i++;
//can we change the order of the tests?
\end{lstlisting}

Time complexity: $O(n)$ comparisons
\vspace{6mm}

The worst case occurs when the target is not found or the target is the last element, the while loop is run \textit{n} times to search through the whole array before returning.

\pagebreak

\section{Binary search}

In fact, we can do better. Yet this requires the array to be \textbf{sorted} beforehand.

Every time we can cut the array in half, and compare the target with the middle element (x[m]). 
\vspace{6mm}

Let's say the segment that the target might be in is x[i..j) (including i and excluding j). We try to maintain that structure that the right hand index is always excluded, and the left hand index is always included, to avoid complications of code.
\vspace{6mm}

If x[m] $<$ target, we know all elements with index $<=$ m are also smaller than the target, because the array is sorted. We can set the lower boundary (i) to m+1.

If x[m] $>$ target, we know all elements with index $>=$ m are also larger than the target, because the array is sorted. We can then set the upper boundary (j) to m. (excluding m)
\vspace{6mm}

In either case, around half of the array is eliminated from consideration.

Process continues until the target is found (x[m] = target).
\vspace{6mm}

\includegraphics[width=13cm]{images/ch7-binarysearch1.png}
\pagebreak

Here is an implementation: \textit{(NOT the one you should follow)}

\begin{lstlisting}
int binarySearch1(int x[], int n, int target){
    int i=0;
    int j=n;
    while(i<j){
        //array segment: x[i..j)
        //stopping condition is i=j because the array would be empty
        int m = (i+j)/2;
        if(x[m]<target){
            //new array segment: x[m+1..j)
            i = m+1;
        }else if(x[m]>target){
            //new array segment: x[i..m)
            j = m;
        }else{
            //a match is found
            return m;
        }
    }
    return -1; //not found
}

int main(){
    int x[] = {1,2,3,3,3,5,7,9};

    cout << binarySearch1(x,8,1) << endl; //0 
    cout << binarySearch1(x,8,2) << endl; //1
    cout << binarySearch1(x,8,7) << endl; //6

    //there are multiple 3s, should it return 2,3 or 4? Is the behaviour consistent?
    cout << binarySearch1(x,8,3) << endl; //4

    //4 is not in the array, can it return something more meaningful other than -1?
    cout << binarySearch1(x,8,4) << endl; //-1
}
\end{lstlisting}

\pagebreak

\subsection*{To be precise}

\textit{Difficult topic}
\vspace{6mm}

How about duplicate targets in the array? To maintain consistency, we should return the location of the leftmost occurrence of the target. Hence, we cannot just stop when a match is found.

How about targets that are not found in the array? We should return the location where the target can be inserted to maintain the order of the array. 

To achieve these goals, we have to change our concept slightly. (Surprisingly these changes lead to neater code)
\vspace{6mm}

We will imagine that we split the array into three sections, x[0..i) contains all elements that are checked $<$ target, x[i..j) contains all elements that are not checked yet, and x[j..n) contains all elements that are checked $>=$ target. 

We maintain this relationship from start to finish. Initially, we set i=0, j=n, so that the first and the last sections are empty signifying all elements are unchecked. Then we gradually check the elements using binary search (similar to the concept explained in the previous section), until i=j, meaning that all elements are checked. Then we return a value.
\vspace{6mm}

Which value shall we return? If the value is present, x[0..i) contains all elements that $<$ target, and x[i..n)\footnote{remember i=j when binary search terminates} contains all elements that $>=$ target. 

If the target is present, it will be at location i. If there are duplicates, the leftmost occurrence will be at location i. If the target is not present. The proper location to insert it is also location i. So i is the correct value to return in all cases. Neat!
\vspace{6mm}

Followed are some illustrations that further elaborate on the concept, and also the implementation in C++.
\vspace{6mm}

\includegraphics[width=12.5cm]{images/ch7-binarysearch71.png}

\includegraphics[width=12.5cm]{images/ch7-binarysearch72.png}

\includegraphics[width=14cm]{images/ch7-binarysearch34.png}

\pagebreak

C++: \textit{(Exercise:  Try to re-implement yourself.)}

\begin{lstlisting}
int binarySearch(int x[], int n, int target){
    int i=0;
    int j=n;
    while(i<j){
        //array segment: x[i..j)
        int m = (i+j)/2;
        if(x[m]<target){
            //new array segment: x[m+1..j)
            i = m+1;
        }else {
            //new array segment: x[i..m)
            j = m;
        }
    }
    return i;
}

int main(){
    int x[] = {1,2,3,3,3,5,7,9};

    cout << binarySearch(x,8,1) << endl; //0 
    cout << binarySearch(x,8,2) << endl; //1
    cout << binarySearch(x,8,7) << endl; //6

    cout << binarySearch(x,8,3) << endl; //2 (leftmost occurance)

    cout << binarySearch(x,8,4) << endl; //5 (the index that number 4 should be inserted)
}
\end{lstlisting}

It might not be completely clear whether the target is in the array, but we could easily check by \texttt{i$<$n \&\& x[i]==target}:

\begin{lstlisting}
int i = binarySearch(x,8,target);
if(i<n && x[i]==target) //Is i<n necessary? Can we change the order of the tests?
    cout << target << " found at location " << i << endl;
else 
    cout << target << " not found, it can be inserted at location " << i << endl;
\end{lstlisting}

Time complexity: $O(\log n)$ comparisons
\vspace{6mm}

In each step, around half of the array is eliminated from consideration, giving a total of $\log_2 n$ steps. Each step takes constant time.

\section{Conclusion}

\begin{table}[h]
    \centering
    \begin{tabular}{|m{6em}|m{9em}|m{18em}|}
        \hline  
        \textbf{Searching Algorithms} & 
        \multicolumn{2}{l|}{Goal: Find location of a target in an array}
        \\ \hline \hline
        
        Algorithm &
        Time Complexity & 
        Remarks
        \\ \hline \hline
        
        Linear search &
        $O(n)$ &
        Works for all arrays
        \\ \hline
        
        Binary search &
        $O(\log n)$ &
        Works only for sorted arrays, much faster
        \\ \hline
    \end{tabular}
\end{table}

Binary search should be used when the data is sorted (which is commonly the case), linear search otherwise.
% TODO: HKOI 2022 SENIOR Q15(?)
\chapter{Sorting Algorithms}

In this chapter we will look at some algorithms that sort an array of integers\footnote{The algorithms also work for other data types with linear order defined. However integer is selected for easier understanding.} of length \textit{n} in ascending order. We will look at how they work and analyze their efficiency (using time complexity). We will only focus on implementation using arrays.

To demonstrate different sorting algorithms, we would use this code right here as a template to verify whether the algorithms work:

\begin{lstlisting}
#include <iostream>
using namespace std;
int main(){
    int x[] = {3,1,4,1,5,9,2,6};
    
    //the sorting function accepts the array and the length of the array, this is common practice, as it is impossible to determine the length of the array in C/C++ when only the array is given.
    sort(x,8); //replace with name of the sorting function

    //print everything in x, with commas properly placed
    //check if the answer is 1,1,2,3,4,5,6,9
    cout << x[0];
    for(int i = 1; i < 8;  i++){
        cout << "," << x[i];
    }
    cout << endl;
}
\end{lstlisting}
\pagebreak
\section{Bubble sort}

Bubble sort swaps the places of two adjacent integers when it realizes they are out of order, until all integers are all sorted.

\includegraphics[width=15cm]{ch8-bubblesort.png}

\pagebreak

% Pseudocode:
% \begin{lstlisting}[language=Python,basicstyle=\rmfamily]
% bsort(x,n):
%     while(not sorted):
%         for j from 0 to n-1:
%             if x[j] > x[j+1]:
%                 swap(x[j],x[j+1]
% \end{lstlisting}

\pagebreak

At the \textit{i}th iteration of the outer loop, the last \textit{i} elements of the array are put at their sorted location. To guarantee the array is sorted, the inner loop is run \textit{n} times.


C++: (\textit{Exercise: Try to re-implement yourself.})
\begin{lstlisting}
void bsort(int x[], int n){
    for(int i = 0; i < n; i++){
        for(int j = 0; j < n-i-1; j++){ //the last i elements are sorted, no need to loop again
            if(x[j]>x[j+1]){
                int temp = x[j];
                x[j] = x[j+1];
                x[j+1] = temp;
            }
        } 
    }
}
\end{lstlisting}

Time complexity: $O(n^2)$ comparisons


There are \textit{n} steps, each step involves checking and swapping data from start to finish, taking time proportional to \textit{n} (in the worst case).

\pagebreak

\section{Selection sort}

Selection sort selects the minimum element and put it in front of the sorted array. Repeats until all elements are extracted.

\includegraphics[width=16cm]{ch8-selectionsort.png}

\pagebreak

% Pseudocode:
% \begin{lstlisting}[language=Python,basicstyle=\rmfamily]
% ssort(x,n):
%     y = []
%     for each element in array x:
%         i = select-minimum(x) //get index of minimum element in x
%         y.append(x[i]) //add this element to the back of the newly created array
%         x.remove(i) //remove that minimum element from consideration
%     return y
% \end{lstlisting}

\pagebreak

C++: (\textit{Exercise: Try to re-implement yourself.})
\begin{lstlisting}
void ssort(int x[], int n){
    for(int i=0; i<n; i++){
    
        //find minimum value and its index in the array
        int minValue = x[i]; //Can we put 0 here instead?
        int minIndex = i;
        for(int j=i; j<n; j++){
            if(minValue > x[j]){
                minValue = x[j];
                minIndex = j;
            }
        }
        
        //swap x[i] with x[minIndex]
        x[minIndex] = x[i];
        x[i] = minValue;        
    }
}
\end{lstlisting}

Time complexity: $O(n^2)$ comparisons


There are \textit{n} steps, each step involves finding the minimum value, taking time proportional to \textit{n}.

\pagebreak

\section{Insertion sort}

Insertion sort assumes the leftmost element is a sorted array, then slowly adds elements in one by one, preserving the order of the sorted array.

\includegraphics[width=15cm]{ch8-insertionsort.png}


% Pseudocode:
% \begin{lstlisting}[language=Python,basicstyle=\rmfamily]
% isort(x,n):
%     for i from 1 to n:
%         insert(x,i,x[i])

% insert(x,i,v):
%     j = i-1
%     while(j>0 and x[j] > v){
%         x[j+1] = x[j]
%         i--
%     }
%     x[j] = v
% \end{lstlisting}
% 
\pagebreak

C++: (\textit{Exercise: Try to re-implement yourself.})
\begin{lstlisting}
void isort(int x[], int n){
    for(int i=1; i<n; i++){
        int j = i-1;
        while(j>=0&&x[j] > x[j+1]){
            int temp = x[j];
            x[j] = x[j+1];
            x[j+1] = temp;
            j--;
        }
    }
}
\end{lstlisting}

Time complexity: $O(n^2)$ comparisons


There are \textit{n} steps, each step involves finding the correct location to insert the value concerned, taking time proportional to \textit{n}.

\pagebreak

\section{Merge sort}

\textit{Difficult topic}


Merge sort first splits the array repeatedly until all arrays are of length 1, where they are sorted without any actions. Then it merges back them two by two in order, eventually we get back the whole array in sorted order.

\includegraphics[width=15cm]{ch8-mergesort.png}

\pagebreak

% Pseudocode:
% \begin{lstlisting}[language=Haskell,basicstyle=\rmfamily]
% msort(x) = merge(msort(l),msort(r))
%     where l = first half of x, r = second half of x
% msort({}) = {}
% msort({a}) = {a} 

% merge(a,b)
%     | null a: b
%     | null b: a
%     | head a <= head b : [head a] ++ merge(tail a, b)
%     | else: [head b] ++ merge(a, tail b)
% \end{lstlisting}
% 

C++: (\textit{Exercise: Try to re-implement yourself.})

Call by \texttt{msort(x,0,8);}

\begin{lstlisting}
void merge(int x[], int a, int m, int b){
    //segment a: x[a..m)
    //segment b: x[m..b)

    //create array c that stores the result of the merge 
    int c[b-a];

    //store pointers of a,b,c to indicate the progress of the merge
    int ai = 0;
    int bi = 0; 
    int ci = 0;
    
    //perform the merge
    while(ai < m-a && bi < b-m){
        if(x[ai+a]<=x[bi+m]){
            c[ci] = x[ai+a];
            ci++;
            ai++;
        }else{
            c[ci] = x[bi+m];
            ci++;
            bi++;
        }
    }

    //one of the segments have completely used up, copy the remaining elements to the end of c
    while(ai < m-a){
        c[ci] = x[ai+a];
        ci++;
        ai++;
    }
    while(bi < b-m){
        c[ci] = x[bi+m];
        ci++;
        bi++;
    }

    //overwrite the data of the original segment with data in c
    for(int i=0;i<b-a;i++){
        x[i+a] = c[i];
    }
}

void msort(int x[], int start, int end){
    if(start+1>=end) return; //base case: the array is empty or with only 1 element
    int mid = (start+end)/2;
    msort(x,start,mid); //msort first half recursively [start..mid)
    msort(x,mid,end); //msort second half recursively [mid..end)
    merge(x,start,mid,end);
}
\end{lstlisting}

Time complexity: $O(n\log n)$ comparisons


There are $\log_2 n$ steps, each step involves function \textit{merge}, taking time proportional to \textit{n}.

\subsection*{A sprinkle of magic (OUT OF SCOPE)}

I know this piece of notes is probably too much for some of you, but as a Computer Science fanatic, I just have to give you something more to see. Well, you see the merge sort code took more than a page, but it can be much shorter in another programming language, Haskell. 

Here is the Haskell implementation:

\begin{lstlisting}[language=Haskell]
msort [] = []
msort [x] = [x]
msort xs = merge (msort ls) (msort rs)
    where (ls,rs) = halve xs

merge [] ys = ys
merge xs [] = xs
merge (x:xs) (y:ys) = if x <= y then x:merge xs (y:ys) else y:merge (x:xs) ys
    
halve xs = (take m xs, drop m xs)
    where m = (length xs) `div` 2

\end{lstlisting}

Of course you won't be able to understand the details of the code, but what I want you to know is that merge sort can be implemented with just 10 lines of code in some other languages. It is not necessarily one and a half page long.

\section{Quicksort}

\textit{Of less importance, difficult topic}


Don't have time to cover, there should be plenty of resources online on this topic.

% C++: (\textit{Exercise: Try to re-implement yourself.})

% Call by \texttt{qsort(x,0,8);}

% \begin{lstlisting}
% int partition(int x[], int l, int r){

%     //set the first element in x[l..r) to be the pivot
%     int p = x[l];
    
%     //refer to illustration on the maintenance of variables i and j
%     int i = l+1;
%     int j = r;
%     while(i<j){
%         if(x[i] < p){
%             i++;
%         }else{
%             j--;
%             int temp = x[i];
%             x[i] = x[j];
%             x[j] = temp;
%         }
%     }
    
%     //swap x[l] and x[i-1], so that the pivot is in the middle
%     x[l] = x[i-1];
%     x[i-1] = p;
    
%     //return location of the pivot
%     return i-1;
% }

% void qsort(int x[], int start, int end){
%     if(start+1 >= end) return; //base case: the array is empty or with only 1 element
%     int k = partition(x,start,end); 
    
%     //the array is now in three parts: 
%     //1. x[start..k) all have values < pivot 
%     //2. x[k] is the pivot
%     //3. x[k+1..end) all have values >= pivot
%     //we recursively sort the first and third parts
%     qsort(x,start,k);
%     qsort(x,k+1,end);
% }
% \end{lstlisting}
\section{Counting sort}

\textit{Of less importance}


Don't have time to cover, there should be plenty of resources online on this topic.

\section{Conclusion}

\begin{table}[h]
    \centering
    \begin{tabular}{|m{6em}|m{9em}|m{18em}|}
        \hline  
        \textbf{Sorting Algorithms} & 
        \multicolumn{2}{l|}{Goal: Sort elements in an array in ascending order}
        \\ \hline \hline
        
        Algorithm &
        Time Complexity & 
        Remarks
        \\ \hline \hline
        
        \makecell[lb]{Bubble sort \\ Selection sort \\ Insertion sort} &
        $O(n^2)$ &
        Only used on short arrays (seldom the case for programming competitions) because of their poor time complexity 
        \\ \hline
        
        \makecell[lb]{Merge sort \\ Quicksort} &
        $O(n\log n)$ &
        Good enough time complexity, without significant disadvantages. Hence they are used most often.
        \\ \hline
        
        Counting sort &
        $O(n)$ &
        Significant disadvantages, limited usage despite its fast time complexity, seldomly used. \tablefootnote{Know more about counting sort: \href{https://www.interviewcake.com/concept/java/counting-sort}{https://www.interviewcake.com/concept/java/counting-sort}}
        \\ \hline
    \end{tabular}
\end{table}

Merge sort and quicksort both have a time complexity of $O(n\log n)$, without significant disadvantages. Hence they are used most often. 


I hope you enjoyed the burst in knowledge, and if you are one of the contestants in the coming HKOI or other programming competitions, I also wish you all the best.
\section*{Exercises}
\addcontentsline{toc}{section}{Exercises}

\begin{questions}

% TODO: HKOI 2022 SENIOR Q9

\miquestion Given an array a[100] of 100 distinct integers. After running the program segment below,
which of the following statements must be true? \footnotecite{hkoi:2022hs}

\begin{lstlisting}
for (i = 0; i <= 99; i++)
    for (j = 0; j <= 99; j++)
        if (a[i] < a[j]) {
            temp = a[i];
            a[i] = a[j];
            a[j] = temp;
        }
\end{lstlisting}

\begin{multiplechoice}
    \item The array is sorted in ascending order
    \item The array is sorted in descending order
    \item \code{a[0]} is the largest element of the array
    \item The array is reversed
\end{multiplechoice}

\label{q:hkoi:2022hs:q10}

\answer{
    \ref{q:hkoi:2022hs:q10} A\\\\
    This algorithm is a non-standard insertion sort implementation. The subarray \code{a[0]...a[i]} must be sorted in ascending order after the i-th iteration.
}
\end{questions}
% \chapter{Programming Practices}

% \section{Logging}

% \section{Dry running/ Code tracing}

% \section{camelCase}

% \section{comments}

% \section{Object Oriented Programming}
\include{conclusions}

%now enable appendix numbering format and include any appendices
\appendix
\include{zappendix1}
\include{zappendix2}
\chapter{A Note to C Programmers}

At first sight, C and C++ programs look very different. 

For example, when you print the same thing above using C, you will do:

\begin{lstlisting}
//C
#include <stdio.h> 
int main(){
    printf("Hello world\n");
}
\end{lstlisting}

To our surprise, we can also do:

\begin{lstlisting}
//C++
#include <cstdio>
using namespace std;
int main(){
    printf("Hello world\n");
}
\end{lstlisting}

For those of you who have learnt C before, I have a good news for you, that your effort is not wasted, as \textbf{you can use all functionality in C program in C++}\footnote{Out of scope: Well, C++ is not exactly a superset of C because there are a small amount of things that can be done by C but not C++, but those things are not a concern at all for students like us}. You can include all C libraries by prepending the name with a 'c', and removing the \texttt{.h}. For instance, \texttt{cmath} and \texttt{ctime}.

It is wise to learn both \texttt{cstdio} and \texttt{iostream}, and to use the appropriate one on suitable occasions in competitive programming competitions like HKOI.

\texttt{printf} is more superior than \texttt{cout} when you want to print some floating point values with a certain number of decimal places. (using \texttt{printf("\%.2f",num);})

While it is easier to get input from a whole line using \texttt{cin.getline} (more in \cref{sec:cingetline}).

\chapter{Solutions for public examination questions}

\begingroup
\parindent 0pt
\parskip 2ex
\def\enotesize{\normalsize}
\def\makeenmark{\relax}
\def\notesname{Solutions}
\theendnotes
\endgroup

% \theanswers
% \ref{q:dse:2016p1:q30} All of them are equivalent

% \ref{q:dse:2016p1:q31} 1234515

% \ref{q:dse:2016p1:q32} 25

%next line adds the Bibliography to the contents page
\addcontentsline{toc}{chapter}{References}
%uncomment next line to change bibliography name to references
\renewcommand{\bibname}{References}
\bibliography{refs}        %use a bibtex bibliography file refs.bib
\bibliographystyle{plain}  %use the plain bibliography style

% \printindex
\end{document}
