\chapter{Setting Up}

Here are some instructions on using C++ on your own machine.

\section{Further resources (Ch 1-3)}
\href{https://www.youtube.com/watch?v=tvC1WCdV1XU&list=PLAE85DE8440AA6B83}{Bucky's C++ Programming Tutorial}\footnote{Link: \href{https://www.youtube.com/watch?v=tvC1WCdV1XU&list=PLAE85DE8440AA6B83}{https://www.youtube.com/watch?v=tvC1WCdV1XU\&list=PLAE85DE8440AA6B83}} (a YouTube playlist) covers most things that you need to know about C++, and also most of the things in this chapter. You only need to watch the first 20-30 videos, as it goes too deep in the later episodes.

\section{Practicing at home}
Practicing is very important. For example, one of the things I would do back then is to remove the one or two lines that I didn't understand in the materials, and see how they affected the program by printing out the values of the variables at different times. 

The two IDEs\footnote{Integrated Development Environment, in short a text editor with tools to make programming easier} I recommend are Code::Blocks and Visual Studio Code. Either one would work.

\subsection*{Code::Blocks}

It is simpler to use, suitable for beginners, but can only be used to write C/C++ code. \href{https://www.codeblocks.org/}{Click here for the official website.}\footnote{Link: \href{https://www.codeblocks.org/}{https://www.codeblocks.org/}}

The first video of \href{https://www.youtube.com/watch?v=tvC1WCdV1XU&list=PLAE85DE8440AA6B83}{Bucky's C++ Programming Tutorial}\footnote{Link: \href{https://www.youtube.com/watch?v=tvC1WCdV1XU&list=PLAE85DE8440AA6B83}{https://www.youtube.com/watch?v=tvC1WCdV1XU\&list=PLAE85DE8440AA6B83}} covers how to use it in detail.

\subsection*{VS Code}

Suitable for students who have experience in using the command line. It is lightweight and works well with other languages. \href{https://code.visualstudio.com/}{Click here for the official website.}\footnote{Link: \href{https://code.visualstudio.com/}{https://code.visualstudio.com/}}

You will have to compile and run the C++ program in the command line (make sure you installed the \href{https://www.youtube.com/watch?v=8CNRX1Bk5sY}{GNG GCC compiler through MinGW})\footnote{Installation tutorial: \href{https://www.youtube.com/watch?v=8CNRX1Bk5sY}{https://www.youtube.com/watch?v=8CNRX1Bk5sY}}.

The commands needed for Git Bash (for Windows users) and the MacOS Terminal are as follows: (may be different for other tools)
\vspace{6mm}

\texttt{g++ -o <executable> <source code>}

\texttt{./<executable>}
\vspace{6mm}

For example,

\texttt{g++ -o test test.cpp}

\texttt{./test}


\section{Structure of a C++ program}
\begin{lstlisting}
//First include the libraries that you are going to use
#include <iostream> 

//A weird line of code that you have to remember every time you write a C++ program.
using namespace std;

int main(){
    cout << "Hello world" << endl;
    return 0;
}
\end{lstlisting}

The main function is the point of entry of the program.

You need to add a semicolon at the end of every statement or else the compiler will shout at you. 

\section{Comparison with C}

At first sight, C and C++ programs look very different.

For example, when you print the same thing above using C, you will do:

\begin{lstlisting}
//C
#include <stdio.h> 
int main(){
    printf("Hello world\n");
}
\end{lstlisting}

To our surprise, we can also do:

\begin{lstlisting}
//C++
#include <cstdio>
using namespace std;
int main(){
    printf("Hello world\n");
}
\end{lstlisting}

For those of you who have learnt C before, I have a good news for you, that your effort is not wasted, as \textbf{you can use all functionality in C program in C++}\footnote{Out of scope: Well, C++ is not exactly a superset of C because there are a small amount of things that can be done by C but not C++, but those things are not a concern at all for students like us}. You can include all C libraries by prepending the name with a 'c', and removing the \texttt{.h}. For instance, \texttt{cmath} and \texttt{ctime}.
\vspace{6mm}

It is wise to learn both \texttt{cstdio} and \texttt{iostream}, and to use the appropriate one on suitable occasions.

\texttt{printf} is more superior than \texttt{cout} when you want to print some floating point values with a certain number of decimal places. (using \texttt{printf("\%.2f",num);})

While it is easier to get input from a whole line using \texttt{cin.getline} (more in \cref{sec:cingetline}).

\section{A note on the choice of programming language}

I would like to stress that the choice of programming language is an arbitary one (that is helpful for those doing the HKOI competition). Most of the concepts in this piece of notes apply to other programming languages such as Java, C and Python.