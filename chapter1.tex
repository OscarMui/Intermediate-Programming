\chapter{Hong Kong Olympiad in Informatics}

Even though the knowledge you will take away by joining the competition is more important than specific skills related to the competition, I think it is still worth it to mention the key points.

\section{About the competition}
\textit{IMPORTANT: rules of HKOI might have changed over the years, please refer to the \href{https://hkoi.org/en/}{HKOI website}\footnote{Link: \href{https://hkoi.org/en/}{https://hkoi.org/en/}} for the latest rules.}
\vspace{6mm}

The competition is split into Junior Group and Senior Group, you would need to be born after a certain date in order to join the Junior Group. (usually S5 students or above have to join the Senior Group, but some lucky ones are young enough for the Junior Group, seize this opportunity if this is your case)
\vspace{6mm}

The competition consists of two parts, heats and finals.
\vspace{6mm}

The heat event is usually held in November, a 1.5-hour written test. Junior Group contestants are given 5 True or False questions, 20 MC questions (each worth 1 mark each), and 20 marks worth of Short Questions. Senior Group contestants are given 25 MC questions (each worth 1 mark each), and 20 marks worth of Short Questions. Making the total score 45. 

There are only two grades you can get in the heats, that is, PASS or FAIL. The mark you get in heats will not affect your award in the finals, The passing mark fluctuates so that around 100 contestants enter the finals. The passing mark of previous years can be found in the \href{https://hkoi.org/en/past-problems/}{official solution of the heat event}.\footnote{Link: \href{https://hkoi.org/en/past-problems/}{https://hkoi.org/en/past-problems/}}
\vspace{6mm}

The final event is usually held in December, a 3-hour practical test. Where you write code to solve 4 problems, each worth 100 marks, bringing the total to 400. Prizes will be given based on the total mark you get out of 400.
\vspace{6mm}

Regardless of your final result, this is a good experience and a good opportunity to learn more about computing, I hope all of you would enjoy it.

\section{General tips on heats}
\begin{enumerate}
    \item Don't be too stressed out, it is fine as long as you get a higher score than the cutoff.
    \item Answer every MC question, make guesses if you are not sure.
    \item Remember basic C++ syntax so that you know the shortest way to write the code needed for the short questions, as there are character limits imposed on your answers. (see the answer sheet of the heat event)
\end{enumerate}

\section{General tips on finals}
\begin{enumerate}
    \item Try some past questions on the \textit{HKOI Online Judge}\footnote{It is a paid service, only allowing purchase through your school. You need to ask your school to give you access} to get a feel of what it feels like coding for three hours straight.
    \item Focus on the first few subtasks of each question. They are usually easier, and you will be able to at least get some of the marks.
    \item Note that you will only get the marks for the subtask when you pass ALL the test cases.
    \item Remember basic C++ syntax to reduce thinking time.
\end{enumerate}

\section{Choice of programming language}

The best programming language for both the heats and finals in my opinion is C++. C++ should be the only language available in the heats starting from year 2022/23. However, you could use other languages like Python in the finals (as long it is listed in the HKOI rules) if you feel more comfortable with it, but you will have to bare the risk that they are only a second class language, meaning that it is not guaranteed that you can get full mark using them (e.g. Python generally runs a bit slower than C++). Though for most of us aiming for the first few subtasks it should be fine.