\chapter{A Note to C Programmers}

At first sight, C and C++ programs look very different. 

For example, when you print the same thing above using C, you will do:

\begin{lstlisting}
//C
#include <stdio.h> 
int main(){
    printf("Hello world\n");
}
\end{lstlisting}

To our surprise, we can also do:

\begin{lstlisting}
//C++
#include <cstdio>
using namespace std;
int main(){
    printf("Hello world\n");
}
\end{lstlisting}

For those of you who have learnt C before, I have a good news for you, that your effort is not wasted, as \textbf{you can use all functionality in C program in C++}\footnote{Out of scope: Well, C++ is not exactly a superset of C because there are a small amount of things that can be done by C but not C++, but those things are not a concern at all for students like us}. You can include all C libraries by prepending the name with a 'c', and removing the \texttt{.h}. For instance, \texttt{cmath} and \texttt{ctime}.

It is wise to learn both \texttt{cstdio} and \texttt{iostream}, and to use the appropriate one on suitable occasions in competitive programming competitions like HKOI.

\texttt{printf} is more superior than \texttt{cout} when you want to print some floating point values with a certain number of decimal places. (using \texttt{printf("\%.2f",num);})

While it is easier to get input from a whole line using \texttt{cin.getline} (more in \cref{sec:cingetline}).
