\section*{Exercises}
\addcontentsline{toc}{section}{Exercises}

\begin{enumerate}[label=\textbf{\thechapter.\arabic*}]

\item Recall the iterative factorial function we defined in Chapter 2, and the recursive version we defined in Chapter 4. 

\begin{lstlisting}
int fact(int x){
    if(x==0) return 1;
    else return x*fact(x-1);
}
\end{lstlisting}

\begin{lstlisting}
int factIter(int x){
    int y = 1;
    for(int i = 1; i <= x; i++){
        y *= i;
    }
    return y;
}
\end{lstlisting}

\begin{enumerate}[leftmargin=0cm]

\item Rewrite both functions in a programming language of your own choice. Test your program to make sure it works. \textit{(Ignore if you do not know other programming languages)}
\item What would be the result of both programs if we input a negative number? If the results are not ideal, suggest a way to improve it.
\item Try calculating \texttt{fact(12)}, \texttt{fact(13)} and \texttt{fact(14)}, are the results correct? If they are not correct, explain why. (If they are correct in your programming language, you are lucky.)
\item What would be the result of both programs if we input large numbers? 

\textit{(Try inputting 1000 to both functions then gradually increase by adding trailing 0s until one of them gives an error. For Python 1000 is enough, for C++ I need 1000000, not sure for other languages)}
\end{enumerate}
\end{enumerate}