\section*{Exercises}
\addcontentsline{toc}{section}{Exercises}

\textit{It is recommended that you first do the questions using pen and paper, before trying to run any code on a computer.}

\begin{questions}

\subsection*{Conditionals and Functions}

\miquestion {\footnotesize [A]} Refer to the following code snippet.

\begin{lstlisting}
int score;
cin >> score;
if(score < 70){
    cout << "A";
}
if(score > 40){
    cout << "B";
}else if(score < 60){
    cout << "C";
}else{
    cout << "D";
}
\end{lstlisting}

Determine the output on the following inputs.

\begin{parts}
\part 30
\part 40
\part 60
\part 70
\end{parts}

\label{q:if1}
\answer{ \ref{q:if1}
\begin{parts}
\part AC
\part AC
\part AB
\part B
\end{parts}
}

\miquestion {\footnotesize DSE\cite{dse:2016p1}} Which of the following pseudocodes produce the same result? 

\begin{enumerate}[label=(\arabic*)]
\item 
\begin{lstlisting}[language=Caml]
if P < 3 and Q > 25 then
    R = R + 1
\end{lstlisting}

\item 
\begin{lstlisting}[language=Caml]
if Q > 25 then
    if P < 3 then
        R = R + 1
\end{lstlisting}

\item 
\begin{lstlisting}[language=Caml]
if P < 3 then
    if Q > 25 then
        R = R + 1
\end{lstlisting}
\end{enumerate}

\begin{multiplechoice}
    \item (1) and (2) only
    \item (1) and (3) only
    \item (2) and (3) only
    \item (1), (2) and (3)
\end{multiplechoice}
\label{q:dse:2016p1:q30}

\answer{
    \ref{q:dse:2016p1:q30} D
    \if\aians1
    In logical operations, such as the ``and'' operator, the order of the conditions does not affect the result.

    Therefore, all three pseudocodes (i), (ii), and (iii) are equivalent and produce the same result. They all check if P is less than 3 and Q is greater than 25, and if both conditions are true, they increment R by 1.
    \fi
}

\miquestion {\footnotesize (UNI)} Consider the general form of an nested if statement with three branches, where cond$_1$ and cond$_2$ are boolean expressions that evaluates to true or false, while body$_1$, body$_2$ and body$_3$ are sequences of statements.

\begin{lstlisting}[language=Caml,mathescape=true]
if cond$_1$ then
    body$_1$
else if cond$_2$ then
    body$_2$
else
    body$_3$
\end{lstlisting}

Describe precisely the conditions when body$_1$, body$_2$ and body$_3$ are executed in terms of cond$_1$ and cond$_2$.

Generalise this to $n-1$ conditions and $n$ bodies.

\miquestion {\footnotesize (Of less importance)[A]} Determine the output of the following code snippets.

\begin{parts}
\item 
\begin{lstlisting}
cout << 17/4 << endl;
\end{lstlisting}

\item 
\begin{lstlisting}
cout << 17.0/4 << endl;
\end{lstlisting}

\item 
\begin{lstlisting}
cout << 17/4.0 << endl;
\end{lstlisting}

\item 
\begin{lstlisting}
double a = 17/4;
cout << a << endl;
\end{lstlisting}

\item 
\begin{lstlisting}
int b = 17.0/4;
cout << b << endl;
\end{lstlisting}

\item 
\begin{lstlisting}
double c = 17.0/4.0;
cout << c << endl;
\end{lstlisting}
\end{parts}

\label{q:mod}
\answer{ \ref{q:mod}

\begin{parts}
\item 4
\item 4.25
\item 4.25
\item 4
\item 4
\item 4.25
\end{parts}
}
\miquestion \textbf{(Leap year).} Write a function that takes in a year and determines whether it is a leap year.

\begin{lstlisting}
bool isLeapYear(int year)
\end{lstlisting}

A year is a leap year if it is a multiple of 4. (e.g. 2016,2020,2024 are leap years) \\
However, a year is not a leap year if it is a multiple of 100, even though it is a multiple of 4. (e.g. 2100, 1900 are not leap years) \\
However, a year is a leap year if it is a multiple of 400, even though it is a multiple of 100. (e.g. 1600, 2000 are leap years)

Do you think your answer is the simplest implementation? Discuss with your instructor.

\subsection*{Loops and Functions}
\miquestion {\footnotesize DSE\cite{dse:2016p1}} What is the output of the following code snippet?

% \begin{lstlisting}
% int S = 0;
% for(int j = 1; j <= 5; j++){
%     cout << j;
%     S += j;
% }
% cout << S;
% \end{lstlisting}

\begin{lstlisting}
S <- 0
for J from 1 to 5
    Output J
    S <- S + J
Output S
\end{lstlisting}

\begin{multiplechoice}
\item 12345
\item 1234515
\item 5432115
\item 5432121
\end{multiplechoice}

\label{q:dse:2016p1:q31}

\answer{
    \ref{q:dse:2016p1:q31} B

    \if\aians1
The for loop iterates from j = 1 to j = 5.
Inside the loop, cout << j; prints the value of j on each iteration, resulting in the output "12345".

Additionally, S += j; adds the value of j to the variable S on each iteration, accumulating the sum of the numbers from 1 to 5.
After the loop, cout << S; prints the value of S, which is the sum of the numbers from 1 to 5, resulting in the output "15".

Therefore, the overall output of the code snippet is "1234515".
    \fi
}

\miquestion {\footnotesize HJ\cite{hkoi:2022hj}} What is the output of the following program?

\begin{lstlisting}
int x, y, z, i;
int main() {
    x = 0;
    y = 0;
    z = 0;
    for (i = 1; i <= 2022; i++) {
        if (i % 10 == 0)
            x++;
        else if (i % 5 == 0)
            y++;
        else
            z++;
    }
    cout << x << ' ' << y << ' ' << z;
return 0;
}
\end{lstlisting}

\begin{multiplechoice}
    \item 202 202 1618
    \item 202 404 1416
    \item 404 202 1618
    \item 404 808 810
\end{multiplechoice}

\label{q:hkoi:2022hj:q7}
\answer{ \ref{q:hkoi:2022hj:q7} 
A

x counts the sum of numbers divisible by 10 in range [1, 2022], in which the answer is
202.
y counts the sum of numbers divisible by 5 but not 10 in range [1, 2022], which is 202
as well.
z counts the rest of numbers, which is 2022-202-202=1618
}

\miquestion Write a program that keeps asking users to input a rating from 1 to 5, but will not terminate until the user enters 5. Different error messages should be shown when the user input is $\leq 0$, between 1 and 4, and $>5$. Assume the user only inputs integers.

Write two versions of the program, one with a do while loop, one without.

Expected input/ output:
\begin{lstlisting}
Give me a rating from one to five: -55
You can only input a positive number.
Give me a rating from one to five: 6
Thanks but you can at most give a rating of 5.
Give me a rating from one to five: 4
I think I deserve a better rating.
Give me a rating from one to five: 5 
Thank you.
//(and terminates)
\end{lstlisting}

\miquestion {\footnotesize(HKOI, UNI)} \textbf{(Prime Factorisation).} Write a function that takes a positive integer $\geq 2$ and print out all its prime factors separated by spaces.

Expected output:
\begin{lstlisting}
//On input 50
2 5 5
//On input 461244
2 2 3 7 17 17 19
\end{lstlisting}

\miquestion {\footnotesize(UNI)} You are required to design a program that translates a weather report from degree Fahrenheit to degree Celsius. There is always exactly 4 inputs, at 7,8,9,10am respectively.

\[C = \frac{{(F - 32) \times 5}}{{9}}\]

Expected input/ output:
\begin{lstlisting}
Input the temperature at 7am: 48.1
The temperature at 7am is 8.94444oC
Input the temperature at 8am: 50.2
The temperature at 8am is 10.1111oC
Input the temperature at 9am: 53.3
The temperature at 9am is 11.8333oC
Input the temperature at 10am: 54.8
The temperature at 10am is 12.6667oC
\end{lstlisting}

I would admit this question is a bit ambiguous and there is no one single correct answer as you are the one that makes the necessary design choices. The focus is to explore different ways we can utilise functions.

\begin{parts}
\part Write the program with the help of a function with the following signature that return the temperature of the argument in Celsius. 

\begin{lstlisting}
double f(double degf)
\end{lstlisting}

\part Write the same program again, but instead with the help of a function with the following signature that prints the temperature and the time on the screen.

\begin{lstlisting}
void g(int time, double degf)
\end{lstlisting}

\part Write the program again, but instead with the help of a function with the following signature that takes in the current time as an argument. It gets the temperature input from the user, and prints the temperature and time time on the screen.

\begin{lstlisting}
void h(int time)
\end{lstlisting}

\part Which implementation is the best in your opinion? Or is there another implementation that is better than all of the above? Discuss with your instructor.
\end{parts}

\subsection*{Loops and Arrays}
\miquestion {\footnotesize DSE\cite{dse:2016p1}} \code{NUM} is an integer array. What is the output of the following algorithm? 

\begin{lstlisting}
K = 0
while K <= 100 do
    NUM[K] = K * K
    K = K + 1
output ( NUM[3] + NUM[4] )
\end{lstlisting}

\begin{multiplechoice}
    \item 7
    \item 25
    \item 49
    \item 100
\end{multiplechoice}
\label{q:dse:2016p1:q32}

\answer{
    \ref{q:dse:2016p1:q32} B

    \if\aians1
    The algorithm you provided initializes an integer array NUM and populates it with the squares of numbers from 0 to 100. It then outputs the sum of NUM[3] and NUM[4].

Let's follow the algorithm step by step:

Initialize K as 0.\\
Enter the while loop.\\
Calculate NUM[K] as K * K and store the result in NUM[K].\\
Increment K by 1.\\
Repeat steps 3-4 until K becomes greater than 100.\\
Exit the while loop.\\
Output the sum of NUM[3] and NUM[4].\\

Given the algorithm, the output can be determined as follows:

The loop runs from K = 0 to K = 100 (inclusive).\\
On each iteration, the square of K is calculated and stored in NUM[K].\\
After the loop finishes, the values of NUM will contain the squares of numbers from 0 to 100.\\
Finally, the output is the sum of NUM[3] and NUM[4].\\

To find the specific output, we need to calculate NUM[3] and NUM[4] based on the algorithm:

NUM[3] is calculated as 3 * 3, which equals 9.\\
NUM[4] is calculated as 4 * 4, which equals 16.\\
Therefore, the output of the algorithm is the sum of NUM[3] and NUM[4], which is 9 + 16 = 25.
    \fi
}

\miquestion {\footnotesize HJ\cite{hkoi:2022hj}} What is the output of the following program?

\begin{lstlisting}
int dx[4] = {1, 2, 3, 4};
int dy[4] = {4, 3, 2, 1};
int x = 101, y = 100;
int i;
int main() {
    for(i = 1; i <= 100; i++) {
        x = x + dx[x % 4];
        y = y + dy[y % 4];
    }
    cout << x << ' ' << y;
    return 0;
}
\end{lstlisting}

\begin{multiplechoice}
    \item 1 4
    \item 499 500
    \item 495 500
    \item 351 350
\end{multiplechoice}

\label{q:hkoi:2022hj:q16}
\answer{ \ref{q:hkoi:2022hj:q16} 
B

Follow the trace table.
i: 1 2 3 4
x\%4: 1 3 3 3
y\%4: 0 0 0 0
x afterward: 103 107 111 115
y afterward: 104 108 112 116
Value of x and y can then be deduced, which is 103+99*4=499 and 100+100*4=500
respectively.
}

\miquestion {\footnotesize HJ\cite{hkoi:2022hj}} What is the output of the following program?

\begin{lstlisting}
int a[8] =
{2, 0, 2, 1, 1, 1, 1, 3};
int x;
int main () {
    x = 0;
    if (a[3] == 2)
        x = x + 1;
    else if (a[6] % 2 == 1)
        x = x + 2;
    if (a[4] == a[7])
        x = x + 3;
    else
        x = x + 4;
    cout << x;
    return 0;
}
\end{lstlisting}

\begin{multiplechoice}
    \item 1
    \item 2
    \item 4
    \item 6
\end{multiplechoice}

\label{q:hkoi:2022hj:q17}
\answer{ \ref{q:hkoi:2022hj:q17} 
D

a[3] = 2: false
a[6] mod 2 = 1: true
a[4] = a[7]: false
Thus x = 2 + 4 = 6
}

\miquestion Here is a generalisation to the code in \cref{sec:forloops}.

\begin{lstlisting}
int n = ...; //length of the array
int x[n] = ...; //the array
int sum = 0;
for(int i = 0; i < n; i++) { 
    sum += x[i];
}
cout << sum << endl; //31
\end{lstlisting}

\begin{parts}
\part Explain the purpose of the code.
\part A student suggested the following alternative code that is supposed to achieve the same purpose. Their instructor thinks it is incorrect. Explain why. Note that the code should work for any array \code{x} of any length with any values. 

\begin{lstlisting}
int n = ...; //length of the array
int x[n] = ...; //the array
int sum = x[0];
for(int i = 1; i < n; i++) {
    sum += x[i];
}
cout << sum << endl; 
\end{lstlisting}
\end{parts}

\label{q:sumx0}

\miquestion A student is asked to write a program to find the maximum element of an integer array. Here is their attempt:

\begin{lstlisting}
int n = ...; //length of the array
int x[n] = ...; //the array
int maximum = 0;
for(int i = 1; i < n; i++) {
    if(maximum <= x[i]) maximum = x[i];
}
cout << sum << endl; 
\end{lstlisting}

Again, their instructor thinks that it is incorrect. Explain why and write a correct version of the code. Note that the code should work for any array \code{x} of any length with any values. 

\miquestion Write a code snippet that finds the smallest and second smallest element of an array.

\begin{lstlisting}
int n = ...; //length of the array
int x[n] = ...; //the array
int minimum; //after your code finishes, minimum holds the smallest element
int minimum2nd; //after your code finishes, minimum2 holds the second smallest element
// your code goes here
\end{lstlisting}

\miquestion Rewrite both the code snippets in \cref{sec:loopbreak} and \cref{sec:loopcontinue} without using \code{break} and \code{continue}. 

\subsection*{Nested Loops}

\miquestion {\footnotesize HJ\cite{hkoi:2022hj}} Suppose a is an array of four 32 bit signed integers \code{int
a[4]}. Is the following two program segments always output equivalently? 

\begin{lstlisting}
for (w = 1; w <= a[0]; w++)
    for (x = 1; x <= a[1]; x++)
        for (y = 1; y <= a[2]; y++)
        for (z = 1; z <= a[3]; z++)
            cout << "HKOI" << endl;
\end{lstlisting}

\begin{lstlisting}
for (i = 0; i <= 3; i++)
    for (j = 1; j <= a[i]; j++)
        cout << "HKOI" << endl;
\end{lstlisting}

\label{q:hkoi:2022hj:q2}
\answer{ \ref{q:hkoi:2022hj:q2}
The first program prints ``HKOI'' for a[0]*a[1]*a[2]*a[3] times, but the second
does a[0]+a[1]+a[2]+a[3] times, which may not be equal. The statement is false.
}
\miquestion Create a function that takes in a non-negative number $n$ and prints the following pattern on the screen.

\begin{lstlisting}
1
1 2
1 2 3
.............
1 2 3 4 ... n
\end{lstlisting}

\miquestion {\footnotesize(HKOI, UNI)} \textbf{(Pascal Triangle).} Create a function that takes in a non-negative number $n$ and prints the first $n+1$ lines of the required pattern on the screen. We start counting column and row numbers from 0.

Expected output: $(n = 5)$
\begin{lstlisting}
1
1 1
1 2 1
1 3 3 1
1 4 6 4 1
1 5 10 10 5 1
\end{lstlisting}

The first and last entries of each line are 1. (i.e. $\Delta(j,0) = 1$ and $\Delta(j,j) = 1$ for all $j$)

The other $(i,j)$th entries are given by the sum of $(i-1,j-1)$th and $(i-1,j)$th entries. (i.e. $\Delta(i,j) = \Delta(i-1,j-1) + \Delta(i-1,j)$ for all $i,j$ with $0 < j < i$)

How much memory is used? Can we do better? Discuss with your tutor.
\end{questions}