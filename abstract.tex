\chapter*{Preface}

This piece of notes is originally made for students who are going to join the Hong Kong Olympiad in Informatics (HKOI). It covers some topics in the \link{https://hkoi.org/en/competition-syllabus/}{HKOI syllabus} that I found the most useful. Aiming to introduce senior secondary school (high school) students with more advanced programming theories, so as to prepare them for competitive programming competitions like HKOI. 

At the same time, it also serves as a good revision material for students taking public examination on programming, such as HKDSE ICT and IB Computer Science. However, it is highly likely that not everything in the syllabus is covered. Some advanced contents are included, especially in the exercises, to enlighten students who aim to study Computer Science in university. 

The name of this piece of notes is ``Intermediate Programming'', so I expect readers to have basic knowledge in programming, including if statements, variables, loops, arrays and functions, in any programming language. However, we will still briefly go through them in \cref{sec:elementary} as a recap and for you to familiarise yourself with the C++ syntax.

It is a simplified piece of notes with a huge number of links to other resources, as the internet is a better teacher than me, yet I am here to provide you with information that I found the most useful when I was in your position a few years ago.

The source code of this notes can be found on \link{https://github.com/OscarMui/Intermediate-Programming-Notes}{GitHub}.

\section*{About me}

I am a third-year undergraduate student studying Computer Science at the University of Oxford, United Kingdom. I joined the senior group HKOI before in the year 2020-2021 and obtained a silver award at the finals. 

\section*{Choice of programming language}
I would like to stress that the choice of programming language is an arbitrary one (that is helpful for those doing the HKOI competition). Most of the concepts in this piece of notes apply to other programming languages such as Java, C and Python.

\section*{A word of warning}

This piece of notes aims to include everything in the shortest amount of time possible, so explanations and examples may be inadequate. Unfortunately, a small amount of sections of the notes are incomplete due to a lack of time, you should be able to fill in the gaps on your own by asking Google.

The fact that it serves multiple parties means that some of the sections may not be at a suitable difficulty. The \textit{Difficult topic}, \textit{Of less importance} and \textit{Out of scope} labels may help. Readers should consult their instructors when they are not sure which parts they should skip, or which exercises they should do.

Note that questions with wordings like ``Discuss with your instructor'' are open ended so there will not be an absolute correct answer.

