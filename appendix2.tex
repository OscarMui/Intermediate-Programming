\chapter{Setting Up}
\label{sec:settingup}
Here are some instructions on using C++ on your own machine.

Practicing is very important. For example, one of the things I would do back then is to remove the one or two lines that I didn't understand in the materials, and see how they affected the program by printing out the values of the variables at different times. 

The two IDEs\footnote{Integrated Development Environment, in short a text editor with tools to make programming easier} I recommend are Code::Blocks and Visual Studio Code. Either one would work.

\subsection*{Code::Blocks}

It is simpler to use, suitable for beginners, but can only be used to write C/C++ code. \href{https://www.codeblocks.org/}{Click here for the official website.}\footnote{Link: \url{https://www.codeblocks.org/}}

The first video of \href{https://www.youtube.com/watch?v=tvC1WCdV1XU&list=PLAE85DE8440AA6B83}{Bucky's C++ Programming Tutorial}\footnote{Link: \url{https://www.youtube.com/watch?v=tvC1WCdV1XU&list=PLAE85DE8440AA6B83}} covers how to use it in detail.

\subsection*{VS Code}

Suitable for students who have experience in using the command line. It is lightweight and works well with other languages. \href{https://code.visualstudio.com/}{Click here for the official website.}\footnote{Link: \url{https://code.visualstudio.com/}}

You will have to compile and run the C++ program in the command line (make sure you installed the \href{https://www.youtube.com/watch?v=8CNRX1Bk5sY}{GNG GCC compiler through MinGW})\footnote{Installation tutorial: \url{https://www.youtube.com/watch?v=8CNRX1Bk5sY}}.

The commands needed for Git Bash (for Windows users) and the MacOS Terminal are as follows: (may be different for other tools)

\texttt{g++ -o <executable> <source code>}\\
\texttt{./<executable>}

For example,

\texttt{g++ -o test test.cpp}\\
\texttt{./test}
